\documentclass{article}
\usepackage[margin=2cm,bottom=2cm]{geometry}
\usepackage{comment}
\usepackage[utf8]{inputenc}
\usepackage{graphicx}
\usepackage{mathtools}
\usepackage[normalem]{ulem}
\usepackage{setspace}
\usepackage{MnSymbol}
\usepackage{soul}
\usepackage{amsmath}

\begin{document}
\title{COMP/CMPE 314 - Principles of Programming Languages - Notes}
\author{Chris Stephenson, Istanbul Bilgi University, Department of Mathematics, and course students}
\maketitle

\section*{Memory Management}
\begin{flushleft}
 In Java we can allocate memory with \underline{new} keyword.
 \begin{verbatim}
  int[] a = new int[n];
 \end{verbatim}
 In C/C++ we have \underline{malloc} and \underline{free}.\\
 \bigskip
 If we have \underline{manual} memory management;
 \begin{itemize}
  \item[(a)] We can have a memory leak. (forget to free)
  \item[(b)] We can have a catastrophe. (free memory still in use)
 \end{itemize}
 Even this is not cheap.\\
 Especially if we think about \underline{fragmentation}.
 \begin{verbatim}
  +------------------------------------------------+
  |      | free |      | free |      | free |      |
  +------------------------------------------------+
 \end{verbatim}
 Manual is not a good idea. - Leads to problems in practice.\\
 We want an automatic system.\\
 It needs to detect when allocated memory can be handed back.
 \begin{itemize}
  \item We want a system that is \underline{sound}. (does not hand back free memory)
  \item We want a system that is \underline{complete}. (hands back all free memory)
 \end{itemize}
\end{flushleft}

\subsection*{Solution 1 (PHP, Swift)}
\subsubsection*{Reference Counts}
 \begin{flushleft}
  Increase for each new reference to a piece of memory,\\
  decrease when a reference is deleted or overwritten.\\
  When the count drops to \underline{zero} hand back the memory.
  \begin{verbatim}
   class x {
      .. x a = this;    <-- Circular Reference
   }
  \end{verbatim}
  In this example, there is no way to fall zero! It causes memory leak.
 \end{flushleft}
 \pagebreak

\subsection*{Solution 2 (Java, C\#, Racket, Haskell, ..)}
\subsubsection*{Garbage Collection}
\begin{flushleft}
\begin{itemize}
 \item[] Find the memory in use starting from base references,
 \item[] Mark the memory in use,
 \item[] Free all the rest.
\end{itemize}
\bigskip
\textbf{Cheney (1970)}\\
 \begin{itemize}
  \item[] Divide memory into two,
  \item[] Use A for allocating memory,
  \item[] When A is ``exhausted"
  \item[] Copy referenced memory from A to B
  \item[] Mark the old copies with ''broken hearts"
 \end{itemize}
\bigskip
\textbf{Exploit}\\
\begin{itemize}
 \item Short term allocation
 \item Long term allocation
\end{itemize}
\end{flushleft}

\subsection*{Generational Garbage Collection}

\pagebreak
\begin{verbatim}
 (define (substitute [to-sub : lambda] [for : symbol] [in : lambda]) : lambda
      (type-case lambda in
           [lambda-id (id) (if (symbol=? for id) to-sub in)]
           [lambda-apply (lhs rhs) (lambda-apply (substitute to-sub for lhs)
                                                 (substitute to-sub for rhs))]
           [lambda-def (bound body) (if (symbol=? for bound) in
                                    (if (or (not (set-member for (free-identifiers body)))
                                            (not (set-member bound (free-identifiers to-sub)))
                                    (lambda-def bound (substitute to-sub in body)))))]
\end{verbatim}
\end{document}
