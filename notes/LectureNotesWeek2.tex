\documentclass{article}
\usepackage[margin=2cm,bottom=2cm]{geometry}
\usepackage{hyperref}
\usepackage{comment}
\usepackage[utf8]{inputenc}
\usepackage{graphicx}
\usepackage{mathtools}
\usepackage[normalem]{ulem}
\usepackage{color}
\usepackage{setspace}
\usepackage{MnSymbol}

\newcommand\tab[1][1cm]{\hspace*{#1}}
\DeclareMathSizes{10}{10}{10}{10}

\begin{document}
\title{COMP/CMPE 314 - Principles of Programming Languages - Notes}
\author{Chris Stephenson, Istanbul Bilgi University, Department of Mathematics, and course students}
\maketitle

\section*{Functions/Methods}
\begin{flushleft}
Robert's on Methods;
\begin{itemize}
 \item "Hiding complexity"
 \item "Tools for programmers"
 \item "Method calls as expressions"
 \item "Method calls as messages"
\end{itemize}
\sout{"The calling mechanism is that the actual parameters are copied to the formal parameters and the code is executed."}
\emph{\color{red} it's a lie!}\\
\doublespacing
\section*{How methods are used?}
\tab{Within a function/method you can call another method (We have certainly done that! Example: \verb|Math.min()|)}\\
{\color{red}Missing:} You can call a method in the expression for the actual parameters;\\
\end{flushleft}
\center{\verb|myMethod(Math.max(a, b));| $\leftarrow$ Piping methods}\\
\begin{flushleft}
$\star$ Linking codes is very powerful.\\
\doublespacing
\begin{itemize}
 \item Used for decomposition - Break a problem into pieces
 \item Used for algorithms - Abstraction\\
 (Example: Binary search)\\
\end{itemize}
\end{flushleft}
\begin{flushleft}
\centerline{\noindent\rule{18cm}{0.4pt}}
\tab{In the below java code which is taken from "The Art and Science of java" lets calculate the following equation:}\\
\center{$\binom{n}{r} = \frac{n!}{(n-r)! r!}$}\\
\begin{verbatim}
private int factorial(int n) {
  int result = 1;
  for (int i = 1; i <= n; i++) {
    result *= i;
  }
  return result;
}

public int combinations(int n, int k) {
  return factorial(n) / (factorial(k) * factorial(n - k));
}
\end{verbatim}
\end{flushleft}

\begin{flushleft}
\tab{If we try to calculate \verb|combinations(15, 2)| java will return \verb|-4| which is incorrect and there is no error information. It should return 105. This is because of the max integer limits.}\\
\doublespacing
The calculation should be done like this;
\begin{verbatim}
public int computeBinomialCoefficient(int n, int k) {
  int ans = 1;
  k = Math.min(k, n - k);
  for (int i = 0; i < k; i++) {
    ans = ans * (n - k + 1 + i);
    ans = ans / (i + 1);
  }
  return ans;
}
\end{verbatim}
\centerline{\noindent\rule{18cm}{0.4pt}}
\tab{We wrote \underline{a calculator} to make a programming language, we need to add \underline{abstraction}. (= generalisation. In place of numbers we will use \underline{identifiers} in our calculations.)}\\
Example: 
\begin{verbatim}
 int square(int x){ // Here x is formal parameter and bound identifier
  return x * x;
 }
\end{verbatim}
$\vdots$\\
\underline{application}
\begin{verbatim}
 println(" " + square(7)); // Here "7" is the actual parameter
\end{verbatim}
\tab{We evaluate the body of the function by substituting the \underline{actual} parameter for the formal parameter in the body of the function.}\\
\tab{We need to extend our data definition.}\\
\tab{We need to add;}\\
\begin{itemize}
 \item Function definition
 \item Function application
\end{itemize}
\tab{To start with we will keep our function definitions seperate.}\\
\tab{Our functions will have a seperate \underline{namespace}.}\\
\tab{Our interpreter will have two inputs - a list of function definitions - as an expression.}\\
(Which may include function applications)



\section*{What does a function have?}
\begin{itemize}
 \item A formal parameter - name (identifier)
 \item A body - expression (in our extended version)
 \item A name - identifier (in the function name namespace)
 \item Function name - identifier in the function name namespace
 \item Actual parameter - identifier
\end{itemize}
\tab{Extend our expression data definition.}\\
\tab{We need to add two more variants.}\\
\begin{itemize}
 \item Function application
 \item Identifier
\end{itemize}
\tab{We need to extend the interpreter to deal with the two new variants in the data definition.}
\begin{itemize}
 \item Identifier $\rightarrow$ error
 \item Function application\\
 \verb|(apply (find-function function-list function-name) actual-parameter)|
\end{itemize}
\tab{Statement of purpose for apply, substitute the actual parameter for the formal parameter everywhere in the body of the function, then evaluate the result.}

\begin{verbatim}
 (define '(apply function-definition actual-parameter){
  (interp (subst (function-definition formal-parameter) actual-parameters))
 })
\end{verbatim}
\pagebreak
\textbf{What does subst do?}\\
name expression expression $\rightarrow$ expression

\underline{Template:}\\
number $\rightarrow$ number\\
arith-expression $\rightarrow$ same expression but subst the operands\\
identifier $\rightarrow$ if the identifier is the identifier to be substitued, we replace it\\
function application $\rightarrow$ function application, but subst the actual parameter\\
\doublespacing
\textbf{The syntax of the $\lambda$-calculus:}\\
$\Lambda \rightarrow \vartheta$ \\
$\Lambda \rightarrow$ ($\lambda \vartheta \lambda$)\\
$\Lambda \rightarrow$ ($\Lambda \Lambda$)\\
\textbf{Valid sentences:}\\
$\mathit{a}$\\
($\mathit{a}$ $\mathit{b}$)\\
($\lambda$ $\mathit{x}$ $\mathit{x}$) - informally this is the identity function\\
\sout{($\mathit{a}$ $\mathit{b}$ $\mathit{c}$)} - \textit{not valid}\\
($\lambda$ $\mathit{f}$ ($\lambda$ $\mathit{x}$ ($\mathit{f}$ ($\mathit{f}$ $\mathit{x}$)))) - a function doubler\\
($\lambda$ $\mathit{v}$ ($\mathit{...}$))
\end{flushleft}
\end{document}