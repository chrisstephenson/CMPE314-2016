\documentclass{article}
\usepackage[margin=2cm,bottom=2cm]{geometry}
\usepackage{hyperref}
\usepackage{comment}
\usepackage[utf8]{inputenc}
\usepackage{graphicx}
\usepackage{mathtools}
\usepackage[normalem]{ulem}
\usepackage{color}
\usepackage{setspace}
\usepackage{MnSymbol}
\usepackage[T1]{fontenc}
\usepackage{soul}

\newcommand\tab[1][1cm]{\hspace*{#1}}
\DeclareMathSizes{10}{10}{10}{10}

\begin{document}
\title{COMP/CMPE 314 - Principles of Programming Languages - Notes}
\author{Chris Stephenson, Istanbul Bilgi University, Department of Mathematics, and course students}
\maketitle

\section*{The "real" world}
\begin{flushleft}
\textbf{"Evaluation \underline{Strategies}"}\\
\begin{itemize}
 \item a strange word to use?
 \item we make the languages.
\end{itemize}
The problem is that most languages have design defects, so evaluating them is confusing.\\
\bigskip
\textbf{What do we do?}\\
When and What and How parameters are evaluated?
\end{flushleft}

\subsection{When?}
In Java;
\begin{verbatim}
  myfunc(a, g(b), c + d);
  myFunc(a++, a++, a++);	// order matters!
\end{verbatim}
** Java compiler (javac) starts evaluating from \underline{left} however C compiler (gcc) starts from \underline{right}.**\\
\bigskip

\textbf{Are the parameters evaluated left to right or right to left?}\\
In Java, left to right\\
In C, \underline{UNDEFINED!}? (Different compilers different answers)


\begin{flushleft}
\textbf{\underline{OR}} We cannot evaluate the parameters at all. And simply pass the expressions to the function body. (All programming languages do this for \underline{some} things)\\
Racket is a \underline{strict} language, which means parameters \underline{are} evaluated.
\end{flushleft}

\begin{verbatim}
;#lang lazy
#lang racket
(define (my-if condition t f)
  (cond
    (condition t)
    (else f)))
    
(define (fac n)
  (my-if (< n 1) 1 (* n (fac (- n 1)))))
\end{verbatim}
Here \verb|my-if| looks like a function. But it isn't a function in racket. It tries to calculate parameters first. Thus it leads to infinitive loop.\\
If we uncomment \verb|#lang lazy| it will work. 

\begin{flushleft}
\underline{Except} for \underline{if}\\
In Java \verb|f(a) && g(b)| is a boolean \st{expression}\\
They call it "Short circuited evaluation"\\
Evaluated left to right and stop when we know the answer.
\end{flushleft}

\begin{flushleft}
 
\underline{A practical example:}
\begin{verbatim}
  int i = 0;
  while(i < a.length && a[i] != 0) {
    ...
    i++;
  }
  
  while(a[i]!=0 && i<a.length){	// after a swap, what goes wrong?
    ...
    i++;
  }
\end{verbatim}
If the condition goes to \verb|i > a.length|, the \verb|a[i]| throws an error. So the order is important!
\end{flushleft}

\begin{flushleft}
Is evaluation in this language "Strict?" "Non Strict?" "Sometimes Strict?"\\
A different way of binding formal parameters: \noindent\rule{3cm}{0.4pt}
\end{flushleft}
\begin{center}
\textbf{\underline{Environments}}
\end{center}

\begin{flushleft}
\begin{itemize}
 \item Substitution is not how the real world works (For reasons of efficiency)
 \item An environment is an ordered store of identifier/value pairs
 \item For function application we add formal parameter/actual parameter value pairs to the environment and use that environment to evaluate the function body.
 \item Our \underline{evaluator} takes the environment as an additional parameter when the evaluator evaluates an identifier, it looks up the value in the environment.
\end{itemize}

\end{flushleft}
\subsection*{Data Definitions}
An environment is empty \underline{OR} an identifier/value pair plus an environment.
\begin{itemize}
\item look up Env. environment identifier $\rightarrow$ value or \underline{Error}
\item extend Env. environment identifier $\rightarrow$ environment
\end{itemize}
\begin{verbatim}
Example: 
(fundef myFunc (a b c) (+ a (* b c))) //function definition

(myFunc 3 7 (+ 2 9)) //function application
\end{verbatim}

\subsubsection*{Evaluate a function application}
\begin{verbatim}
(eval (extend Env (extend Env (extend Env (empty Env, 'a, 3) 'b, 7) 'c, 11) '(+ a (* b c))))

(fundef f1 x (f2 4))
(fundef f2 y (+ x y))
\end{verbatim}
Whats wrong with f2?\\
It's a free identifier!\\
In Java:
\begin{verbatim}
int f2(int y){
  return x + y;
}
\end{verbatim}
evaluate (f1 42)\\
 x gets bound to 42\\
 y gets bound to 4\\
 (+ x y) looks up these values in the environment\\
 The answer is 46 \textbf{Wrong!!}\\
\bigskip
Identifier capture!\\
This is called "dynamic scope" and it is a bug.
\bigskip
\begin{flushleft}
\underline{The solution} (for now)\\
When we apply a function, do \underline{not} extend the \underline{current environment} with formal parameter/value bindings. Extend the \underline{empty} environment.
\end{flushleft}

\subsubsection*{Grammar}
\begin{flushleft}
$1)$ $\Lambda\rightarrow\upsilon$

$2)$ $\Lambda\rightarrow(\lambda$ 
$\upsilon$
$\Lambda)$

$3)$ $\Lambda\rightarrow(\Lambda$
$\Lambda)$

Data structure and therefore definitions and programs will reflect this.\\
\bigskip
M[x:=N]\\
\end{flushleft}

\begin{flushleft}
$1)$ M is an identifier\\
\tab{a) M=x $\Rightarrow$ M[x	:=N] is N}\\
\tab{b) M$\neq$x $\Rightarrow$ M[x:=N] is M}\\
\bigskip
$2)$ M is an application (just do the two parts)\\
\bigskip
$3)$ M is a function definition\\
\tab{Case (a) is easy $\lambda$yM1, y=x}\\
\tab{Case (b) is hard $\lambda$yM1, y$\neq$x}\\
\end{flushleft}

\subsubsection*{Church Numerals}
\begin{flushleft}
n $\equiv$ ($\lambda$ f ($\lambda$ x (f (f (f...x)))))\\
$\tab{\tab{\tab{\overline{n f's}}}}$
\end{flushleft}

\begin{flushleft}
ChurchB(n) $\equiv$ (f (ChurchB(n -1)))\\
ChurchB(1) $\equiv$ ($\lambda$ f ($\lambda$ x (f x)))\\
\bigskip
Church(n) = ($\lambda$ f (x x (ChurchB(n))))\\
\end{flushleft}

\begin{flushleft}
TWO=($\lambda$ f ($\lambda$ x (f (f x))))\\
ZERO=($\lambda$ f ($\lambda$ x x))\\
\end{flushleft}
\bigskip
\begin{flushleft}
\textbf{A successor function "SUCC"}\\
SUCC=($\lambda$ n ($\lambda$ f ($\lambda$ x (f ((n f) x))))\\
onSUCC=($\lambda$ n ($\lambda$ f ($\lambda$ x ((n f) (f x))))
\end{flushleft}

\begin{flushleft}
To add m and n;\\
($\lambda$ m ($\lambda$ n ((m succ) n)))
\end{flushleft}


\section{Version 2 of these notes - need merging by hand!}

\section{Evaluation Strategies}

\newline A strange word to use 
\newline We make the languages. 
\newline


The problem is that host languages have design defects, so evaluating them is confusing
\newline \newline
When and What and How parameters evaluated?
\newline \newline
When ?
\begin{enumerate}
    \item myFunc (a, b, c+d)
    \begin{enumerate}
        \item expr1 = a
        \item expr2 = b
        \item expr3 = c+d
    \end{enumerate}
\end{enumerate}
\newline
Evaluation Order Matters\newline
\newline
myFunc (a++, a++, a++), order matters in such cases.
\newline \newline
Are the parameters "left to right" or "right to left" ? 
\newline \newline
OR  We can not evaluate parameters at all and simply pass expressions to function body.
(All programming languages do this for some things.)
\newline \newline
Racket is a strict language which means parameters evaluated.
\newline \newline
(define (fact n ))
 (if (< n 1) 1 (* n (fact (- n 1))))
\newline \newline
********************************************
\newline
(define (my-if condition t f) 
 (cond
  (condition t)
  (else f)
 )
)
\newline
\newline
(define (fact n ))
 (my-if (< n 1) 1 (* n (fact (- n 1))))
 \newline
 This is Wrong Evaluation !
 \newline
 \newline
 It Works on \# lang lazy 
 \newline 
 \newline
 output : 
 \newline
 \newline
 (fact 7) 
 \newline
 5040 
 \newline 
 \newline
 In Java f(a) \&\& g(b) is not a boolean expression.
 \newline \newline
 They call it "short circuited evaluation". 
 It is evaluated left to right and stop when we know the answer.
 \newline
 \newline
\newline 
A Practical Example :
\newline
\newline 
While ( i < a.length \&\& a[i]!= 0)\{ \\
    ..... i++; .... \\
    \} \\
\newline
\newline
What is wrong with next code template ?
\newline \newline
int i = 0;
\newline
While ( a[i]!= 0 \&\& i < a.length)\{ \\ 
    ..... i++; .... \\
    \}
\newline
\newline

It gives Array Out Of Bounds Exception.
\newline
\newline
Is Evaluation Strict or Not ? Sometimes Strict ?
\newline
\newline
It is Sometimes Mostly Strict.
\newpage
\section{Environments}

 Substition is not how Real World works. \\
 (For real reasons of efficiency.) \\
\newline
\newline
Our evaluation works the environment as additional parameter. When the evaluation evaluates an identifier, it looks up the value in the environment.
\newline
\newline
\section{Data Definition}
An environment is empty OR in identifier/ value pair plus an environment.
\newline
\newline
look up Env identifier -- value or Error
\newline
\newline
extend Env environment identifier value -- environment
\newline
\newline
(fundef myfunc (a b c)\\
(+ a (* bc))) \\
function definition
\newline
\newline
Example 
\newline
(myfunc 3 7 (+ 2 9))\\
function application
\newline
\newline
An evaluator takes the environment as additional parameter when the evaluator evaluates an identifier it looks up the value in the environment.
\newline
\newline
(eval(extend Env (extend Env (empty Env, 'a, 3) 'b, 7) 'c, 11) (+ a (* b c)))
\newline
\newline
(fundef f1 x (f2 4))   x -- acts like global environment \\     free identifiers
\newline
\newline
(fundef f2 y (+x y))    identifier capture   \\
 int f2 ( int y ) \} Broken Invalid \\ 
 (f1 (f2 3)) \} wrong
 \\
\\
this is called "dynamic scope" Dynamic Scope is a bug
\newline
\newline
evaluate (f1, f2) - x gets bound to 42, y gets bound to 4
\newline
\newline
(+ x y) looks up those values in the environment - The answer is 46 Wrong !
\newline
\newline
The Solution (for now)-- When we apply a function, do not extend the current environment with formal parameter / value bindings. Extends the empty environment.
\newline
\newline
\newline
\newline
M [x := N]
\begin{enumerate}
    \item M is an identifier 
    \begin{enumerate}
    \item M = x  then M[x:=N] is N
    \item M = x  then M[x:=N] is M
    \end{enumerate}
    \item M is an application just do the two parts 
    \item M is a function definition
\end{enumerate}
\newline
\newline
Case (a) is easy λyM1 , y = x \\
Case (b) is hard λyM1 , y != x (not equal)
\newline
\newline


\end{document}

