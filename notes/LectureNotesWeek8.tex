\documentclass{article}
%\usepackage{textgreek}

\title{CMPE314 Notes08 Haskell - Functional Programming \& Sugaring $\lambda$ Calculus}
\date{April 2016}

\begin{document}

\maketitle


\section{Haskell}
\begin{flushleft}
Haskell is: \\-Purely functional\\ -Statically typed \\ - Lazy(entirely )\\
\end{flushleft}


\begin{flushleft}
 In addition to this, Haskell is:\\ -Very interesting language \\ - Become important commercial language \\ - Facebook works on it \\ - Almost pure  $\lambda$ Calculus (Closer to  $\lambda$ calculus than Racket
\end{flushleft}


\vspace{2mm}

\begin{flushleft} Haskell is not for production and other tales(youtube video)\\
\usepackage{https://www.youtube.com/watch?v=mlTO510zO78} \\
\end{flushleft}


\doublespacing
\section{Functional Programming}
\begin{flushleft}
Functions are first class values.\\
They have no state nor side effects.\\
Guarantee that if parameters are same, the result is the same.\\

\vspace{2mm}
\textbf{How can we do this in our interpereter?}

\subsection*{1) Types}
A function application will no longer be (name expr)\\
It will be (expr expr)\\
This means we have more than one type of expression.\\
So expression values are no longer automatically numbers.\\
So we have run time types. So we can have booleans.\\

Now, in our new language expression can be whether function, boolean or number.\\

\subsection*{2) How will functions be named?}
Just like any other expression.\\
Identifiers( which are always formal parameters) and are then applied to actual parameters.\\
This will also be true for functions.\\

\vspace{5mm}
\textbf{What would be like to have?}


(let
   
  \hspace{1cm}(myfunc ($\lambda$ x ( + x 3 )))\\
  \hspace{1cm}(myfunc2 ($\lambda$ y (+ y 4 )))\\
   
  \hspace{1cm}\textless program \textgreater
   \\)

\vspace{5mm}

\textbf{After desugaring:}
\vspace{2mm}

((  $\lambda$ (myfunc myfunc2)\\
\textless program \textgreater)\\
\hspace{1cm}( $\lambda$ (x) (+ x 3) )\\
\hspace{1cm}( $\lambda$ (y) (+ y 4) ))\\

\textbf{\underline {Let} is \underline{sugar.}}





\subsection*{3) Closure }
(let
   
  \hspace{1cm}((multiply-list-by-n ($\lambda$  (l n)\\
  \hspace{1cm} (map ($\lambda$ (x) (* x n))) l))\\

  \hspace{1cm}((multiply-list-by-n '(4 5) 3)\\
  \hspace{1cm}((multiply-list-by-n '(5 6) 7)\\
  
 \vspace{5mm}
 We need to distinguish function definitions from function values when we evaluate a function definition, we get a function value. A function value is a function definition plus the environment at the point where the functiıon value was evaluated. This is called a closure.


\subsection*{4) When we apply a function value?}

Extend the environment in the function value with the bindings of the formal parameters to the actual parameters

\vspace{2mm}
\textbf{Sugaring the $\lambda$-calculus; }\\
\vspace{2mm}
Some definitions;

($F^{0}$M)  $\equiv$ M\\
($F^{n\neq1}$M)  $\equiv$ F(($F^{n}$M)\\
\vspace{8mm}





\hspace{1cm} Church Numeral = $C_n$\\
\vspace{1mm}
\hspace{1cm}  $C_n$ $\equiv$($\lambda$ f ($\lambda$ x ( $f^{n}$ x )))\\
\hspace{1cm}  $C_0$ $\equiv$($\lambda$ f ($\lambda$ x ( $f^{0}$ x )))  $\equiv$ ( $\lambda$ f ( $\lambda$ x x )) \\
\hspace{1cm}  $C_1$ $\equiv$($\lambda$ f ($\lambda$ x ( $f^{1}$ x )))  $\equiv$ ( $\lambda$ f ( $\lambda$ f x )) \\
\hspace{1cm}  $C_2$ $\equiv$($\lambda$ f ($\lambda$ x ( $f^{2}$ x )))  $\equiv$( $\lambda$ f ( $\lambda$ x f (f x ))) \\
\vspace{5mm}

\hspace{1cm}\underline{Let}- as we did for our interpreter

\hspace{1cm}\textbf{Succ} ($\lambda$ n ( $\lambda$ f ( $\lambda$ x) f (( n f) x )))\\
\hspace{1cm}\textbf{True} ( $\lambda$ x ( $\lambda$ y x ))\\
\hspace{1cm}\textbf{False} ( $\lambda$ x ( $\lambda$ y y ))\\
\hspace{1cm}\textbf{If} ( $\lambda$ a ( $\lambda$ b ($\lambda$ c ((a b ) c))))\\
\hspace{1cm}\textbf{Zero?} ( $\lambda$ n (( n ($\lambda$ x FALSE)) TRUE))\\
\vspace{5mm}





\section{Data Structure}

\hspace{5mm} If we can make a pair, we can make a data structure.\\
\vspace{2mm}
\hspace{5mm} What are the semantics of CONS FIRST REST (in Racket)?\\
\vspace{1mm}
\hspace{5mm} (FIRST ( CONS A B )) $\equiv$ A

\hspace{5mm} (REST (CONS A B ))  $\equiv$ B

\vspace{2mm}
\hspace{6mm}\textbf{CONS:} ($\lambda$ a ($\lambda$ b ( $\lambda$ ((f a ) b)))) \\
\vspace{1mm}
\hspace{6mm}\textbf{FIRST:} ($\lambda$ c (c TRUE))\\
\vspace{1mm}
\hspace{6mm}\textbf{REST:} ($\lambda$ c (c FALSE))\\

\end{flushleft}



\end{document}