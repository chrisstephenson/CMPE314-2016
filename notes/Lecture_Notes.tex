\documentclass{article}
\usepackage[margin=2cm,bottom=2cm]{geometry}
\usepackage{hyperref}
\usepackage{comment}
\usepackage[utf8]{inputenc}
\usepackage{graphicx}
\usepackage{mathtools}
\usepackage{color}
\newcommand\tab[1][1cm]{\hspace*{#1}}
\usepackage{amsmath}
\usepackage{array}
\usepackage{setspace}
\usepackage{soul}
\usepackage{amssymb}
\usepackage{tcolorbox}
\usepackage{alltt}
\usepackage{listings}


\begin{document}
\title{COMP/CMPE 314 - Principles of Programming Languages - Notes}
\author{Chris Stephenson, Istanbul Bilgi University, Department of Mathematics, and course students}
\maketitle

\begin{flushleft}
\textcolor{blue}{\textbf{{\huge Week 1 Lecture Notes}}}
\section*{Quiz \#1 Answers}
\subsection*{Question\#1}
\subsubsection*{Language:}
  \begin{itemize}
        \item A finite alphabet
        \item A set of strings of the symbols in the alphabet (usually infinitive)
   \end{itemize}
 
\subsubsection*{A Grammar:}
  \begin{itemize}
   \item A set of productions (rewriting rules)
   \item A finite alphabet of terminal symbols
   \item A finite set of non-terminal symbols\\
\vspace*{0.2cm}
\item[] \tab{\LARGE $\bullet$ $\rightarrow$ $\bullet$ }\\
\bigskip
(Something goes to something)
   \item A sentence symbol S
  \end{itemize}

[a] // S $\rightarrow$ a // S $\rightarrow$ aS
  \begin{itemize}

   \item $S \rightarrow number$\\
   $S \rightarrow S + S$\\
   $S \rightarrow S * S$\\
   $S \rightarrow (S)$\\
   alphabet = [+, *, (), number]
  
  \end{itemize}
  

\subsection*{Question\#2}

  \begin{verbatim}
    (define (interp [a : ArithC]) : number
        (type-case ArithC a
            [numC (n) n]
            [plusC (l r) (+ (numC-n l) (numC-n r))]
            [multC (l r) (* (numC-n l) (numC-n r))]))
  \end{verbatim}
  
  \begin{verbatim}
    In this code;
    (+ 4 3) works but,
    (+ (* 3 7) (+ 4 2)) crashes
  \end{verbatim}
  
  When we convert numC to interp, it turn to work.

  \begin{verbatim}
    (define (interp [a : ArithC]) : number
        (type-case ArithC a
            [numC (n) n]
            [plusC (l r) (+ (interp l) (interp r))]
            [multC (l r) (* (interp l) (interp r))]))
  \end{verbatim}
  
  
\section*{Programming Languages}
\begin{tabular}{c c}
 \textbf{Programming Languages} & \textbf{Kinds of Programming Languages}\\
 \begin{tabular}{ l c r }
    Java & Assembly & Swift \\
    Python & C & Prolog \\
    Ruby & C\# & Fortran \\
    Objective-C & Racket & R \\
    Lisp & Haskell & Whitespace \\
    Javascript & Pascal & Giuseppe \\
    Scala & HTML & Ada \\
    Pick & Perl & XML \\
    SQL & BASH & XSDL \\
    Scratch &  &  \\
  \end{tabular}
  &
  \begin{tabular}{ c }
    Markup Language \\
    Functional \\
    Object oriented \\
    Procedural \\
    Scripting \\
    Graphical \\
    Experimental \\
    Declarative \\
    Machine \\
  \end{tabular}
\end{tabular}


\begin{flushleft}
The code below is valid in Java and also valid in C. \\
What value does the function/method \verb|funny2| returns?

\begin{verbatim}
 int funny(int a, int b){
    return a + 2 * b;
 }
 
 int funny2(){
    int a = 2;
    return funny(a++, a++);
 }
\end{verbatim}

In this example the value returned in Java and C are different. Java returns 8 but C returns 7.\\
This situation caused by the compilers of this languages. In \verb|funny(a++, a++)| statement Java starts from left \verb|a++| but C starts from right \verb|a++|.\linebreak

Let us consider this now\\
1000000000 + 2000000000\\
"1000000000" + "2000000000"\\
\bigskip
For first one Java return negative. Because it is accepted numbers as int. And maximum int value is 2,147,483,647.\\
In the second one Java concatenates the two strings like "10000002000000" (But some languages adds them)\\
\vspace*{0.5cm}
\begin{verbatim}
int not an Integer // Integer a box in which is not an Integer
                      BigInt - It is not big and it is not an int
\end{verbatim}
\pagebreak
\subsection*{\underline{Syntactic Sugar}}
\begin{flushleft}
\begin{verbatim}
Loops --> while       |   for(<init>; <continue>; <steps>){
          Do while    |      <body>
          for         |    }
          foreach     |     <init>
                      |        while(<continue>){
                      |           <body>
                      |         <step>
                      |   }

\end{verbatim}
\end{flushleft}
\begin{flushleft}
All programming languages are data formats for data input to other programs.\\
\bigskip
\boxed{Java}\boxed{Byte Code}\\
\hspace{0.6cm}\boxed{Code}\tab\boxed{C}\boxed{Machine Code}\\
\hspace{3cm}\boxed{C}\\
\vspace*{0.6cm}
\hspace{0.3cm}Bootstrapping $\rightarrow$ self compilation
\vspace*{1cm}\\
Program text $\xrightarrow{parse}$ Data Structure $\xrightarrow{interp/eval}$ Answer\\
\end{flushleft}
\begin{verbatim}
 (eval (parse '(+ 2 (* 3 4))))
\end{verbatim}
Parsing is common to all compilers/interpreters
\end{flushleft}
\end{flushleft}
\begin{flushleft}
\noindent\makebox[\linewidth]{\rule{\paperwidth}{0.4pt}}\\
\vspace*{1.5cm}
\textcolor{blue}{\textbf{{\huge Week 2 Lecture Notes}}}
\section*{Function/Methods}
\subsubsection*{Roberts on methods:}
  \begin{itemize}
        \item "Hiding complexity"
		\item "Tools for programmers"
		\item "Method calls as expressions"
		\item "Method calls as messages"
   \end{itemize}
 
\begin{flushleft}
The calling mechanism is that the actual parameters are copied to the formal parameters and the code is executed.
\emph{\color{red} it's a lie!}\\

\end{flushleft}
\subsection*{How methods are used}
Within a method you can call another method.(We have certainly done that $\rightarrow$ example Math-min)

\begin{flushleft}
$\Rightarrow {\color{red}Missing:}$
You can call a method in the expression for the actual parameters. $\rightarrow$ myMethod(Math.max(a,b));
\end{flushleft}
\begin{flushleft}
 Used for decomposition - Break a problem into pieces
 \end{flushleft} 
\begin{flushleft}
Used for algorithms - abstraction (example binary search)
\end{flushleft}


\newcommand*{\Comb}[2]{{}^{#1}C_{#2}}%

\textbf{\begin{center}
$\binom nr=\frac{n!}{(n-r)! r!}$
$=\frac{15!}{(15-2)!2!}= 105$
\end{center}}

We wrote \underline{a calculator} to make a programming language, we need to add \underline{abstraction}. (= generalisation. In place of numbers we will use \underline{identifiers} in our calculations.)
\begin{flushleft}
Example: 
\end{flushleft}
\begin{verbatim}
 int square(int x){ 
  return x * x;
 }
\end{verbatim}
$Note:$ x is formal parameter and bound identifier


\begin{flushleft}
\underline{application}
\end{flushleft}
\begin{verbatim}
 println(" " + square(7));
\end{verbatim}
$Note:$ "7" is the actual parameter

\begin{flushleft}
\tab{We evaluate the body of the function by substituting the \underline{actual} parameter for the formal parameter in the body of the function.}\\
\end{flushleft}
\tab{We need to extend our data definition.}\\
\tab{We need to add;}\\
\begin{itemize}
 \item Function definition
 \item Function application
\end{itemize}
\tab{To start with we will keep our function definitions seperate.}\\
\tab{Our functions will have a seperate \underline{namespace}.}\\
\tab{Our interpreter will have two inputs - a list of function definitions - an expression.}\\
(Which may include function applications)
  
\subsection*{What does a function have?}
\begin{itemize}
 \item A formal parameter - name (identifier)
 \item A body - expression (in our extended version)
 \item A name - identifier (in the function name namespace)
 \item Function name - identifier in the function name namespace
 \item Actual parameter - identifier
\end{itemize}
\tab{Extend our expression data definition.}\\
\tab{We need to add two more variants.}\\
\begin{itemize}
 \item Function application
 \item Identifier
\end{itemize}
\tab{We need to extend the interpreter to deal with the two new variants in the data definition.}
\begin{itemize}
 \item Identifier $\rightarrow$ error
 \item Function application\\
 \verb|(apply (find-function function-list function-name) actual-parameter)|
\end{itemize}
\tab{Statement of purpose for apply, substitute the actual parameter for the formal parameter everywhere in the body of the function, then evaluate the result.}

\begin{verbatim}
 (define '(apply function-definition actual-parameter){
  (interp (subst (function-definition formal-parameter) actual-parameters))
 })
\end{verbatim}



\begin{flushleft}
\textbf{What does subst do?}\\
\end{flushleft}
name expression expression $\rightarrow$ expression

\begin{flushleft}
\underline{Template:}\\
\end{flushleft}
\begin{flushleft}
number $\rightarrow$ number\\
arith-expression $\rightarrow$ same expression but subst the operands\\
identifier $\rightarrow$ if the identifier is the identifier to be substitued, we replace it\\
function application $\rightarrow$ function application, but subst the actual parameter\\
\end{flushleft}
\textbf{Valid sentences:}\\
$\mathit{a}$\\
($\mathit{a}$ $\mathit{b}$)\\
($\lambda$ $\mathit{x}$ $\mathit{x}$) - informally this is the identity function\\
{($\mathit{a}$ $\mathit{b}$ $\mathit{c}$)} - {not valid}

($\lambda$ $\mathit{f}$ ($\lambda$ $\mathit{x}$ ($\mathit{f}$ 
($\mathit{f}$ $\mathit{x}$)))) - a function doubler\\
\end{flushleft}

\begin{flushleft}
\noindent\makebox[\linewidth]{\rule{\paperwidth}{0.4pt}}\\
\vspace*{1.5cm}
\textcolor{blue}{\textbf{{\huge Week 3 Lecture Notes}}}
\subsubsection*{A program that looks the same but evaluates to two different answers answers should worry us.}
  \begin{itemize}
        \item "Integrated Development Environment"
		\item "Unintegrated Development Environment"
   \end{itemize}


\begin{verbatim} 
 int funny(int a, int b){ 
  return 2 * a + b;
 }
 
 int haha(){
  int z = 2;
  return funny(z++, z++);
}
\end{verbatim}
%

\center z++ $\rightarrow$ side effect\
 
\begin{flushleft}
The ++ post operator produces a value the same as the value of its operand. But it changes the subsequent value of the operand as a \emph{\color{red} side effect //alligator}\\
\end{flushleft}

\begin{flushleft}
In a program using a for loop to process an array we have unnecessary(maybe) stale.
\end{flushleft}

\begin{flushleft}
Look at Hodoop(big data) $\rightharpoonup$ map reduce (must be stateless)
\end{flushleft}\

\center text(parse)$\rightarrow$ first rep(desugar)$\rightarrow$ second rep(eval)$\rightarrow$ evaluation\


\center "desugaring" = "macro processing"\

\begin{flushleft}
Example:
\end{flushleft}

\begin{flushleft}
In Racket, cond is designed into a series of nested if statements.\\
What does our program evaluation look like ?
$(eval(desugar(parse '(* (-3)(+ 7 8)))))$
\end{flushleft}


\begin{flushleft}
The syntax of the $\lambda$-calculus (my idiosyncrotic version)
\end{flushleft}

\begin{flushleft}
\center
$\Lambda\rightarrow\upsilon$\
\center
$\Lambda\rightarrow(\lambda$ 
$\upsilon$
$\Lambda)$\
\center
$\Lambda\rightarrow(\Lambda$
$\Lambda)$
\
\end{flushleft}

\begin{flushleft}
\subsubsection*{Informal meaning}
\end{flushleft}
\begin{flushleft}

$\Lambda\rightarrow\upsilon\Longrightarrow$identifier

$\Lambda\rightarrow(\lambda$ 
$\upsilon$
$\Lambda)\Longrightarrow$anonymous function definition

$\Lambda\rightarrow(\Lambda$
$\Lambda)\Longrightarrow$function application
\end{flushleft}


\begin{verbatim} 
 int funny(int y){ 
  return y;
 } 
\end{verbatim}
%
\begin{flushleft}
(($\Lambda$ x x)(a b)) // x is bond and a,b are free
\end{flushleft}

\begin{flushleft}
A "complete" program will have no "free" identifiers\\
A part of that program may have free identifiers
\end{flushleft}

\begin{flushleft}
If M is an identifier, x $FI(M)={x}$\\
If M is an application (M1,M2) $FI(M)=FI(M1) U FI(M2)$\\
If M is an definition ($\Lambda$ x M1)
\end{flushleft}

\begin{flushleft}
FI[($\lambda$ x ($\lambda$ y x)]={}\\
FI[($\lambda$ y x)]={x}
\end{flushleft}


\begin{flushleft}
\subsubsection*{Informally}
T= ($\lambda$ x ($\lambda$ y x)) \\
(T a) $\rightarrow$ ($\lambda$ y a) // direction with $\beta$\\
((T a) b) $\rightarrow$ a // direction with $\beta$
\end{flushleft}

\begin{flushleft}
F=($\lambda$ x ($\lambda$ y y))\\
((F a) b) $\rightarrow$ b // direction with $\beta$
\end{flushleft}

\end{flushleft}



\begin{flushleft}
\noindent\makebox[\linewidth]{\rule{\paperwidth}{0.4pt}}\\
\vspace*{1.5cm}
\textcolor{blue}{\textbf{{\huge Week 4 Lecture Notes}}}
\section*{Lies and Equality (In Java and other languages)}
"=" does not mean "equals" \\
"\&" does not mean "and"\\
"$||$" does not mean "or"\\   
"!" does not mean "surprise"\\
int does not mean Integer\\
BigInteger does not mean BigInteger\\              
\end{flushleft}

\begin{flushleft}
We use == for "equals"\\
we use \&\& for logical and\\
We use $||$ for logical or\\
\end{flushleft}

\doublespacing
\begin{flushleft}
It means a 32 bit two's complement quality (now most compilers are 64 bit)
\end{flushleft}

\begin{verbatim}
If I declare MyClass a;
a is not necessarily an object
of class MyClass.
\end{verbatim}
%

\subsubsection*{Equality What does == mean ?}
    
\begin{verbatim}
class Equality {
public static void main (String[] args) {
  MyInteger a = new MyInteger(42);
  MyInteger b = new MyInteger(42);
  MyInteger c = new MyInteger(43);
  MyInteger d = c;
  String s1 = "Chris";
  String s2 = "Chris";
  System.out.println("\n a = " + a + ",b = " + b + "\n");
  System.out.println("\n a == b is " + (a == b) + "\n");
  System.out.println("\n s1 == s2 is " + (s1 == s2) + "\n");
  System.out.println("\nd = " + d + "\n");
  System.out.println("\n c == d is " + (c == d) + "\n");
  c.setV(44);
  System.out.println("\n c = " + c +  "\n");
 }
}

class MyInteger {
  int value;
  MyInteger (int v) {
    value = v;
  }
  public String toString () {
    return "" + value;
  }
  public void setV (int v) {
    value = v;
  }
 }
\end{verbatim}    

\doublespacing   
\begin{flushleft}
  It does not mean 'equal in value'(necessarily) 
  \end{flushleft}  
  
  \begin{verbatim}
  String a= "Chris";
  String b= "Chris";
  
  Strings are immutable
  \end{verbatim}
   
    
\begin{flushleft}
Our program
\begin{verbatim}
'((fundef factorial n (if2 n 1(* n (factorial (- n 1))))) (factoaial 3))
\end{verbatim}
\end{flushleft}
$\downarrow$
\begin{verbatim}
parse + desugar
  fdc[name:symbol][arg:symbol][body:s-expression]
  (fdc 'factorial 'n (if(idC n)(numC 1))(numC 1))(multC(numC n)
  				(appC factorial (sub C(idC n)(num 1)
  				
  		(appC 'factorial (numC 3))
\end{verbatim}

\textbf{(if2C \st{(idC 'n)}(numC 1)(mulC \st{(idC 'n)}(appC 'factorial(subC \st{(idC 'n)}(numC 1)))\\
(idc 'n) $\rightarrow$ (numC 3)
}\\


\[ \left( \begin{array}{ccc}
remember & to \\
multiply & by & 3 \\
\end{array} \right)\] 

\[ \left( \begin{array}{ccc}
remember & to \\
multiply & by & 2 \\
\end{array} \right)\]

\[ \left( \begin{array}{ccc}
remember & to \\
multiply & by & 1 \\
\end{array} \right)\] \begin{center}
$\downarrow$
\end{center}

\begin{center}
(mulC \st{(idC 'n)}(appC 'factorial(subC \st{(idC 'n)}(numC 1)))\\
(idC 'n)$\rightarrow$(numC 3)
\end{center}


\subsection*{Transformation}
\begin{flushleft}
($\lambda$ x M) $\equiv$ ($\lambda$ y M [x:=y])\\
$\rightarrow$ [x:=y ]$\rightarrow$ substitude y for x troughout M
\end{flushleft}

\textbf{\begin{flushleft}
$\curvearrowleft$ transformation(optional)
\end{flushleft}}


\begin{flushleft}
($\lambda$ x (f x)) $\overrightarrow{\curvearrowleft}$ f\\
(F y) $\rightarrow$ (f y)\\
(F z) $\rightarrow$ (f z)\\
(F w) $\rightarrow$ (f w)\\
\end{flushleft}

\textbf{\begin{flushleft}
$\beta$ transformation
\end{flushleft}}

\begin{flushleft}
(($\lambda$ x M) y) $\overrightarrow{\beta}$ M [x:=N]\\ 
($\lambda$ x ($\lambda$ y (y x))) $\equiv$ F
\end{flushleft}

\begin{flushleft}
Example \\
((\underline{F M)} N) $\overrightarrow{\beta}$ ((\underline{$\lambda$ y (y M)})N)\\
................ $\overrightarrow{\beta}$ (N M)
\end{flushleft}

\begin{flushleft}
\underline{Some example: } Let M=y N=x \\
((F y) x) $\overrightarrow{\beta}$ (($\lambda$ y (y \underline{y})) x)  // identifier capture has happened (y)\\
................ $\overrightarrow{\beta}$ \underline{(x x)} // disaster !
\end{flushleft}
\pagebreak

\begin{flushleft}
\textcolor{blue}{\textbf{{\huge Week 5 Lecture Notes}}}
\section*{The "real" word}
\end{flushleft}
\begin{flushleft}
"Evalation \underline{Strategies}"\\
a strange word to use?\\
we make the languages.     
\end{flushleft}

\doublespacing
\begin{flushleft}
When and what and How parameters are evalueted.
\end{flushleft}

\doublespacing
\begin{flushleft}
\underline{When?}\\
In Java: $\rightarrow$
\begin{verbatim}
a,g(b),c+d
myfunc(a,b,c);
myFunc(a++,a++,a++); // order matters
\end{verbatim}
\end{flushleft}

\begin{flushleft}
Are the parameters evaluated left to right or r to l ?\\
In Java left to r\\
In C \underline{undefined}? (Different compilers different answers)
\end{flushleft}

\doublespacing
\begin{flushleft}
\underline{OR} We cannot evaluate the parameters at all. And simply pass the expressions to the function body.\\
(All programming languages do this for \underline{some} things)\\
Racket is a \underline{strict} language, which means parameters are evaluated 
\end{flushleft}

\begin{verbatim}
;#lang lazy
#lang racket
(define (my-if condition t f)
  (cond
    (condition t)
    (else f)))
(define (fac n)
  (my-if (< n 1) 1 (* n (fac (- n 1)))))
\end{verbatim}
%

\begin{flushleft}
\underline{Except} for \underline{if}\\
In Java f(a) \&\& g(b) is a boolean expression\\
They call it "Short circuited evaluation"\\
\color{red}Evaluated left to right and stop when we know the answer
\end{flushleft}

\doublespacing
\begin{flushleft}
\underline{A practical example:}
\end{flushleft}

\begin{verbatim}
int i=0;
	while(i<a.length && a[i]!=0)
	{
	...
	i++;
	}
	while(a[i]!=0 && i<a.length) // after swap, what goes wrong?
\end{verbatim}

\begin{flushleft}
Is evaluation in this language "Strict? Non Strict? Sometimes Strict\\
A different way of binding formal parameters: ..........
\end{flushleft}
\begin{center}
\textbf{\underline{Environments}}
\end{center}

\begin{flushleft}
Substitution is not how the real word works (For reasons of efficency)\\
An environment is an ordered store of identifier / value pairs.\\
For function application we add formal parameter / actual parameter value pairs to the environment and use that environment to evaluate the function body.\\
Our \underline{evaluator} takes the environment is an additional parameters when the evaluator avaluates and identifier, it looks up the value in the environment.
\end{flushleft}

\subsubsection*{Data Definitions}
An environment is empty OR an identifier / value pair plus an environment.
\begin{itemize}
\item look up Enr. environment identifier $\rightarrow$ value or \underline{error}
\item extend Enr. environment identifier $\rightarrow$ environment
\end{itemize}
\begin{verbatim}
(fundef myFunc (a b c) (+ a (* b c))) //function definition
(myFunc 3 7 (+ 2 9)) //function application
\end{verbatim}

\subsubsection*{Evaluate function application}
\begin{verbatim}
(extend Enr(extend Enr(empty Enr, 'a, 3)'b, 7)'c 11)'(+ a (* b c)))
(fundef f1 x(f2 4))
(fundef f2 y (+ x y)) // x is a identifier capture this is called "dynamic scope". Dynamic scope is a bug.

In Java: 
int f2(int y){
	return x+y;
}

evaluate (f1 42) 
- x gets bound to 42
  y gets bound to 4
 (+ x y) looks up these values in the environment
- The answer is 46 wrong!!
\end{verbatim}

\begin{flushleft}
\underline{The solution} (for now)- When we apply a function, do not extend \underline{the current environment} with formal parameter / value bindings. Extend the \underline{empty} environment.
\end{flushleft}

\subsubsection*{Grammer}
\begin{flushleft}

$1)$ $\Lambda\rightarrow\upsilon$

$2)$ $\Lambda\rightarrow(\lambda$ 
$\upsilon$
$\Lambda)$

$3)$ $\Lambda\rightarrow(\Lambda$
$\Lambda)$

Data structure and therefore definitions and programs will reflect this.\\
M[x:=N]\\
\end{flushleft}

\begin{flushleft}
$1)$ M is an identifier\\
a) M=x $\Rightarrow$ M[x	:=N] is N\\
b) M$\neq$x $\Rightarrow$ M[x:=N] is N\\
\end{flushleft}

\begin{flushleft}
$2)$ M is an application just do the two parts\\
\end{flushleft}

\begin{flushleft}
$3)$ M is an function definition\\
Case a is easy $\lambda$yM1, y=x\\
Case b is hard $\lambda$yM1, y$\neq$x\\
\end{flushleft}

\subsubsection*{Church Numerals}
\begin{flushleft}
n $\equiv$ ($\lambda$ f ($\lambda$ x (f (f (f...x))))) //nf's
\end{flushleft}

\begin{flushleft}
ChurchB(n) $\equiv$ (f (ChurchB(n -1)))\\
ChurchB(1) $\equiv$ ($\lambda$ f ($\lambda$ x (f x)))\\
Church(n) $\equiv$ ($\lambda$ f (x x (ChurchB(n))))\\
\end{flushleft}

\begin{flushleft}
Two=($\lambda$ f ($\lambda$ x (f (f x))))\\
Zero=($\lambda$ f ($\lambda$ x x))\\
\end{flushleft}
	
\subsubsection*{A successor function "Succ"}
\begin{flushleft}
Succ=($\lambda$ n ($\lambda$ f ($\lambda$ x (f ((n f) x))))\\
onSucc=($\lambda$ n ($\lambda$ f ($\lambda$ x ((n f) (f x))))
\end{flushleft}

\begin{flushleft}
To add m and n ($\lambda$ m ($\lambda$ n ((m succ) n)))
\end{flushleft}
\noindent\makebox[\linewidth]{\rule{\paperwidth}{0.4pt}}\\
\vspace*{1.5cm}
\begin{flushleft}
\textcolor{blue}{\textbf{{\huge Week 7 Lecture Notes}}}
\section*{Evaluation "Strategies"}
\end{flushleft}
\begin{flushleft}
\underline{All about function application}- which is the central concepts in programming \underline{(IMHO)} an \underline{oxymoron.} 
\end{flushleft}

\doublespacing
\begin{flushleft}
- We looked at sequence (2 weeks ago)\\
- Strict vs non-strict\\
- Eager vs Lazy\\
- Left to Right is Right to Left\\
\end{flushleft}

\doublespacing
\begin{flushleft}
What is the passed when we apply a function to actual parameters ?
\end{flushleft}

\doublespacing
\begin{flushleft}
\underline{When?}\\
Example In Java: 
\begin{verbatim}
{ 
a=2;
myFunc(a);
}
------------------------------
void myFunc(int a){
...
a=a+1;
}
\end{verbatim}
\end{flushleft}

\subsection*{What is the value of a?}
\begin{flushleft}
System.out.println("a is" +a); $\leftarrow$ The answer is \underline{2}.
\end{flushleft}

\doublespacing
\begin{flushleft}
\underline{Why?}\\
Because Java / C / C\# / Ruby / Python / C++ are \underline{call by value}\\
For simple types, most common programming languages are call by value.\\
The alternative is \underline{call by preference} (PHP can do this)
\end{flushleft}

\subsection*{But What about complex types ?}
\begin{flushleft}
Examples-array, structure, object.
\begin{verbatim}
{ 
a[ ]=2;
myFunc(a);
System.out.println("a[ ] is" +a); }

void myFunc(int [] a){
---
a[ ]=a[ ]+1
}
\end{verbatim}
\end{flushleft}

\doublespacing
\begin{flushleft}
Call by reference for arrays / structures / Strings / objects 
\begin{verbatim}
[Except C-C structures are naturally call by value]
\end{verbatim}
\begin{flushleft}
Java strings can be considered call-by-value (because Java strings are immutable)
\end{flushleft}
\end{flushleft}

\subsection*{Super accelerated CMPE 313 - (SICP-The Purple Book 1989)}
\begin{flushleft}
What is a List ? A list of numbers ?
\end{flushleft}
\begin{itemize}
\item Empty
\item A pair of a \st{number}(x) and a list of \st{number}(x)
\end{itemize}
\begin{verbatim}
;;add up the numbers in a list
(define (sum-lon lon)
  (cond)
  ((empty? lon) O(empty set))
  (else ..(+ (first lon)...(sum-lon(rest lon)))))
  
(define (mul-lon lon)
  (cond 
  ((empty? lon) 1)
  (else..(* (first lon)(mul-lon(rest lon)))))
  
(define (fold combiner null-value lox)
  (cond
  ((empty? lox) 1)
  (else (* (combiner (first lox) (fold combiner null-value (rest lox))))))
\end{verbatim}

\begin{flushleft}
fold is a powerful \underline{abstraction}.\\
I want to evaluate a \underline{polynomial}.
\end{flushleft}

\begin{center}
${a}_{0} + {a}_{1}x + {a}_{1}{x}^{2} + {a}_{1}{x}^{3} +....+ {a}_{n}{x}^{n}$ 
\end{center}

\begin{flushleft}
Data definition a polynomial is a list of coefficient $({a}_{0},{a}_{1},..,{a}_{n})$\\
Contract eval polynomial number $\rightarrow$ number\\
Horner's rule ${a}_{0} + x({a}_{1} + x({a}_{2} + x({a}_{3}...{a}_{n}))))$\\
To evaluate a polynomial\\
------\\
(define (eval-poly p x)\\
(fold (lambda (l $\gamma$) (+ l (* x r)) $\emptyset$ p))
\end{flushleft}

\begin{flushleft}
\underline{map}\\
(define (map f lox)\\
(fold (lambda (l $\gamma$) (cons (f l) r)) empty lox))))\\
\end{flushleft}

\begin{flushleft}
\underline{filter}\\
(define (filter p? lox)\\
(fold (lambda (L r) (if (p? L) (cons (r) r) empty lox))\\
\end{flushleft}

\doublespacing
\begin{flushleft}
Suppose I am doing a \underline{map} to add 2 to every item in a list
\begin{verbatim}
(define (add2 lon)
  (map (lambda (x) (+ x 2)) lon))
  
(define (mul 7 lon)
  (map (lambda (x) (* x 7)) lon ))
  
(define (carry f x)
  (lambda (y) (f x y)))
---------------------------
(define (arith-map openbar constant)
  (map (carry operator constant) lon))
\end{verbatim}
\end{flushleft}

\begin{flushleft}
\textbf{The $\lambda$-calculus} (Summary)\\
$1)$ $\Lambda\rightarrow\upsilon$

$2)$ $\Lambda\rightarrow(\lambda$ 
$\upsilon$
$\Lambda)$

$3)$ $\Lambda\rightarrow(\Lambda$
$\Lambda)$
\end{flushleft}

\doublespacing
\begin{flushleft}
$\alpha$ ($\lambda$ $\upsilon$ $\Lambda$) $\equiv$ ($\lambda$ $\upsilon$ $\Lambda$ [$\upsilon$:=u])\\
$\gamma$ ($\lambda$ $\upsilon$ (f $\upsilon$)) $\equiv$ f\\
$\beta$ (($\lambda$ $\upsilon$ ${\Lambda}_{1}$) ${\Lambda}_{2}$ $\rightarrow$ ${\Lambda}_{1}$ [$\upsilon$:=${\Lambda}_{2}$]
\end{flushleft}

\begin{flushleft}
Defining substitution is more complicated.\\
$\Lambda$[$\upsilon$:=${\Lambda}_{0}$] - see the slides 
\end{flushleft}

\begin{flushleft}
\textbf{a)} $\Lambda$=u\\
if u=$\upsilon$ $\rightarrow$ ${\Lambda}_{0}$ \\
else $\Lambda$\\
\textbf{b)} $\Lambda$=(${\Lambda}_{1}$ ${\Lambda}_{2}$)\\
(${\Lambda}_{1}$[$\upsilon$=${\Lambda}_{0}$] ${\Lambda}_{2}$[$\upsilon$=${\Lambda}_{0}$]\\
\textbf{c)} $\Lambda$=($\lambda$ u ${\Lambda}_{1}$)\\
(i) u=$\upsilon$ $\rightarrow$ $\Lambda$\\
(ii) u$\neq\upsilon$ if necessary re-letter before substituting in the body.
\end{flushleft}

\subsubsection*{Church Numerals}
\begin{flushleft}
0 $\equiv$ ($\lambda$ f ($\lambda$ x x))\\
1 $\equiv$ ($\lambda$ f ($\lambda$ x (f x)))\\
2 $\equiv$ ($\lambda$ f ($\lambda$ x (f (f x))))
\end{flushleft}

\subsubsection*{Successor}
\begin{flushleft}
Succ=($\lambda$ n ($\lambda$ f ($\lambda$ x (f ((n f) x))))\\
onSucc=($\lambda$ n ($\lambda$ f ($\lambda$ x ((n f) (f x))))\\
(($\lambda$ n ($\lambda$ f ($\lambda$ x (f ((n f) x))))) $\lambda$ f ($\lambda$ x (f x))))\\
$\downarrow \beta$\\
($\lambda$ f ($\lambda$ x (f (($\lambda$ f ($\lambda$ x (f x))) f) x))))\\
$\downarrow \beta$\\
($\lambda$ f ($\lambda$ x (f (($\lambda$ x (f x))) x))))\\
$\downarrow \beta$\\
($\lambda$ f ($\lambda$ x (f (f x)))) $\equiv$ 2!
\end{flushleft}
\begin{flushleft}
\noindent\makebox[\linewidth]{\rule{\paperwidth}{0.4pt}}\\
\vspace*{1.5cm}
\textcolor{blue}{\textbf{{\huge Week 8 Lecture Notes}}}
\section*{Haskell}
\begin{itemize}
\item Purely functional 
\item Statically typed 
\item Lazy.
\end{itemize}
\end{flushleft}

\subsection*{Haskell Curry}
\begin{flushleft}
\begin{itemize}
\item Haskell Platform
\item Simon Reyton-Jones (F\#) (Glasgow) 
\end{itemize}
\end{flushleft}

\section{Functional Programming}
\begin{flushleft}
\begin{itemize}
\item Functions are first class values
\item No state / No side effects\\
\doublespacing
How can we do this in our interpreter ?\\
A function application (name expr) will no longer be, it will be (expr expr);\\
$\downarrow$
\item This means we have more than one \underline{type} of expression
\item So expression values are no longer automatically numbers.
\item So we have runtime types.
\end{itemize}
\end{flushleft}

\subsection*{1- TYPES}
\subsection*{2- How will functions be named ?}
\begin{flushleft}
\begin{itemize}
\item Just like any other expression.
\item Identifiers (which are always formal parameters) and are then applied to actual parameters.
\item This will also be true for functions.
\end{itemize}
\doublespacing
What would be like to have?

(let
   
  \hspace{0.5cm}(myfunc ($\lambda$ x ( + x 3 ))) $\rightarrow$ function def.\\ 
  \hspace{0.5cm}(myfunc2 ($\lambda$ y (+ y 4 ))) $\rightarrow$ function def.\\ 
  \hspace{0.5cm}\textless\ program \textgreater {} $expression$
\\)


\textbf{After desugaring:}

((  $\lambda$ (myfunc myfunc2)\\
\textless\ program \textgreater)\\
\hspace{0.5cm}( $\lambda$ (x) (+ x 3))\\
\hspace{0.5cm}( $\lambda$ (y) (+ y 4)))\\
\textbf{Let is sugar.}\\

(let
   
  \hspace{0.5cm}((double ($\lambda$ (x) ( * x 2 )))\\
  \hspace{0.5cm}(double 7))\\
   
(($\lambda$ (double)\\
\hspace{0.5cm}(double 7))\\
\hspace{0.5cm}($\lambda$ (x) ( * x 2 )))\\
\doublespacing

(let
   
  \hspace{0.5cm}((multiply-list-by-n ($\lambda$ (l n) (map \underline{($\lambda$ (x) (* x n))} l)) // Some function def. Does something different\\
  \hspace{1cm}(multiply-list-by-n '(4 5) 3)\\
  \hspace{1cm}(multiply-list-by-n '(5 6) 7)\\

\end{flushleft}


\subsection*{3- Closure}
\begin{flushleft}
We need to distinguish function definitions from \underline{function values} when we evaluate a function definition, we get a \underline{function value}. \\

\doublespacing
function definition $(one$ $\rightarrow$ $many)$ function values.\\

A function value is a function definition plus the environment at the point where the function value was evaluated. This is called a \textbf {closure.}
\end{flushleft}

\subsection*{4- When we apply a function value?}

Extend the environment in the function value with the bindings of the formal parameters to the actual parameters
\doublespacing
\textbf{Sugaring the $\lambda$-calculus; }\\
\vspace{2mm} Some definitions;

($F^{0}$M)  $\equiv$ M\\
($F^{n+1}$M)  $\equiv$ F(($F^{n}$M)\\

Church Numeral = $C_n$\\
$C_n$ $\equiv$($\lambda$ f ($\lambda$ x ( $f^{n}$ x )))\\
$C_0$ $\equiv$($\lambda$ f ($\lambda$ x ( $f^{0}$ x )))  $\equiv$ ( $\lambda$ f ( $\lambda$ x x )) \\
$C_1$ $\equiv$($\lambda$ f ($\lambda$ x x ( $f^{1}$ x )))  $\equiv$ ( $\lambda$ f ( $\lambda$ x (f x ))) \\
$C_2$ $\equiv$($\lambda$ f ($\lambda$ x x ( $f^{2}$ x )))  $\equiv$( $\lambda$ f ( $\lambda$ x f (f x ))) \\
\doublespacing
\underline{LET} - as we did for our interpreter.\\
\textbf{SUCC} ($\lambda$ n ( $\lambda$ f ( $\lambda$ x) (f (( n f) x ))))\\
\textbf{TRUE} ( $\lambda$ x ( $\lambda$ y x ))\\
\textbf{FALSE} ( $\lambda$ x ( $\lambda$ y y ))\\
\textbf{IF} ( $\lambda$ a ( $\lambda$ b ($\lambda$ c ((a b ) c))))\\
\textbf{ZERO?} ( $\lambda$ n (( n ($\lambda$ x FALSE)) TRUE))\\


\section{Data Structure}

If we can make a pair, we can make a data structure.\\
What are the semantics of CONS FIRST REST (in Racket)?\\
\doublespacing
\hspace{3mm} (FIRST ( CONS A B )) $\equiv$ A // Necessary semantics\\
\hspace{3mm} (REST (CONS A B ))  $\equiv$ B // Necessary semantics\\
\doublespacing
\textbf{CONS:} ($\lambda$ a ($\lambda$ b ( $\lambda$ \textcolor{red}{((f a ) b)})))\\
\textbf{FIRST:} ($\lambda$ c (c TRUE))\\
\textbf{REST:} ($\lambda$ c (c FALSE))\\
\pagebreak
\begin{flushleft}
\textcolor{blue}{\textbf{{\huge Week 9 Lecture Notes}}}
\section*{Laziness}
\begin{flushleft}
\textcolor{red}{\textbf{\underline{Cons:}}} We want it to be lazy. How can we get laziness in an eager language?\\

\doublespacing
($\lambda$ () \underline{$<$expr$>$}) $\rightarrow$ promise\\
\doublespacing
to force our promise to evaluate we just unite (promise)\\
\doublespacing
(delay \underline{(+ 2 3)})\\
\doublespacing
define-syntax-rule makes a function that operates on program text. It is a way of providing syntactic sugar.
\end{flushleft}


\subsection*{Hamming Code}
\begin{flushleft}
Reduce a list of numbers only divisible by the prime numbers 3,5,7 and no other prime number in order 1  3  5  7  9  15  21  25  27  35
\end{flushleft}

\subsection*{Georg Contor}
\begin{flushleft}
The hotel with a infinite number of room is full\\

\doublespacing
What is our interpreter now going to look like?

\doublespacing
We have \underline{function values}. So we need a different way of naming functions

\doublespacing
The only way we now have of creating names is as function parameters

\doublespacing
This is cumbersome to write in the target language.

\doublespacing
Why don't we use the syntactic sugar of a Racket style Let statement?
\begin{verbatim}
(let
  <def list>
    exp)
    ...
   <def list> ---> ((name1 expr1)
                    (name2 expr2)
                    (namen exprn)..)

(let <def list> exp) ---> (let ((name1 expr1))
                              (let ((name2 expr2)) 
                                 (let... expr))))
\end{verbatim}
desugar 2\\
(let (name exp) program) $\xrightarrow{desugar}$ (($\lambda$.name program) expr)
\end{flushleft}

\subsection*{Grammer for our target language}
\begin{flushleft}
S $\rightarrow$ identifier\\
S $\rightarrow$ (S S)\\
S $\rightarrow$ ($\lambda$ identifier S)\\
S $\rightarrow$ (let (identifier S) S)\\
S $\rightarrow$ number\\
S $\rightarrow$ (+ S S)\\

\doublespacing
When we want to multi-parameter functions. We can also do this by desugaring. \\

\doublespacing
($\lambda$ (V1 V2 V3) expr) $\xrightarrow{desugar}$ ($\lambda$ V1 ($\lambda$ V2 ($\lambda$ V3 ... expr))))\\

\doublespacing
(a b c d) $\xrightarrow{desugar}$ (((a b) c) d)
\end{flushleft}


\subsection*{Lambda Calculus}
\begin{flushleft}
((I M) N) $\rightarrow$ (M N)\\
\doublespacing
exponentation = identity function

\doublespacing
The predecessor function is \underline{hard}\\
(PRED n) what we want is n-1\\
Think of a pipeline\\
Let us this transformation an pairs\\
\center
[a b] $\rightarrow$ [b, (Succ B)]
\begin{verbatim}
(lambdan
    (FIRST ((n (lambda c
        ((CONS (REST c)) (SUCC (REST c))))))
        ((CONS ZERO) ZERO))))
        This is a PRED function
\end{verbatim}
\end{flushleft}

\subsection*{Observation}
\begin{flushleft}
($\lambda$ x x) $\equiv$ ($\lambda$ y y) ($\alpha$ transformation)\\
($\lambda$ x ($\lambda$ y y)) $\equiv$ ($\lambda$ a ($\lambda$ b b)) ($\alpha$ transformation)\\
($\lambda$ \textcolor{red}{x} ($\lambda$ y \textcolor{red}{x})) $\neq$ ($\lambda$ a ($\lambda$ \textcolor{red}{b} \textcolor{red}{b}))\\
 \doublespacing
What is the normal form?\\
($\lambda$ \textcolor{red}{x} ($\lambda$ y \textcolor{red}{x})) $\neq$ ($\lambda$ a ($\lambda$ \textcolor{red}{b} \textcolor{red}{b}))\\
What is the essention difference
\end{flushleft}

\subsection*{de Burjin indices}
\begin{flushleft}
($\lambda$ x x) $\rightarrow$ ($\lambda$ 1)\\
($\lambda$ ($\lambda$ 1)) $\equiv$ ($\lambda$ ($\lambda$ 1))\\
($\lambda$ ($\lambda$ 2)) $\neq$ ($\lambda$ ($\lambda$ 2))
\end{flushleft}
\end{flushleft}
\begin{flushleft}
\noindent\makebox[\linewidth]{\rule{\paperwidth}{0.4pt}}\\
\vspace*{1.5cm}
\textcolor{blue}{\textbf{{\huge Week 10 Lecture Notes}}}
\section*{Haskell}
\begin{flushleft}
\underline{(Simon Reyton-Jones)} $\rightarrow$ Glasgow Haskell Compiler new language \underline{F\#}
\begin{itemize}
\item Purely (no assignment, no side effects) Functional-function values are first class value.
\item \underline{Lazy} (example: all lists can be infinite. No special treatment is needed for infinite lists)
\item Sophisticated Type System (That allows generic programming) Variable types.
\item Lots of interesting sugar.
\end{itemize}
\begin{flushleft}
\begin{tcolorbox}
Java is not the solution\\
Java is the problem
\end{tcolorbox}
\bigskip
\end{flushleft}
	
\subsection*{$\beta$ transformation}
\begin{flushleft}
Why did we object to identifier capture?\\
\begin{flushleft}
\doublespacing 
($\lambda$ m (((($\lambda$ x (($\lambda$ y (x y))) m) n))))) y) x)\\ ($\lambda$ m (((($\lambda$ x (($\lambda$ y (x y))) m) n))) y) x)
\end{flushleft}
\textcolor{red}{If we have a identifier capture the result of the B transformations depends on the order they are done in.}
\end{flushleft}
\begin{flushleft}
Is it the case that our definition of B transformation (including the substitution rule) avoids the problem?\\
\bigskip
\end{flushleft}

\subsection*{The church-Rosse property a.k.a the Diamond property}
\begin{flushleft}
If a lambda- sentence M can be transformed by chains of B transformations into sentences M1 and M2, then there exists (possibly empty) chains of B transformations that transform M1 and M2 into some sentence M3.
\end{flushleft}
\begin{flushleft}
A normal form is a lambda - sentence in which no B transformation is possible \\
If Church-Rosser applies every sentence has \underline{at most} \underline{one} normal form.
\end{flushleft}
\begin{flushleft}
\textbf{\underline{Proof}} Suppose M has 2 normal forms, M1 and M2. But Church Rosser tells us that these are chains of B      transformations that transform M1 and M2 into M3. But M1, M2 are normal forms, so these chains must be empty. So M1=M2=M3.
\end{flushleft}
\begin{flushleft}
(($\lambda$ x y) (($\lambda$ x (x x)) ($\lambda$ x (x x))))
\end{flushleft}
\end{flushleft}
\end{flushleft}
\begin{flushleft}
\noindent\makebox[\linewidth]{\rule{\paperwidth}{0.4pt}}\\
\vspace*{1.5cm}
\textcolor{blue}{\textbf{{\huge Week 11 Lecture Notes}}}
\section*{Using Map for Haskell Function}
\begin{flushleft}
Use map to write a Haskell function that multiplies every element of a possibly infinite list of integers by n.\\
\bigskip

{\large $\bigstar$ mulList :: Integer $\rightarrow$ [Integer] $\rightarrow$ [Integer]\\
\vspace*{0.2cm}
$\bigstar$ mulList n l= map ((*) n) \underline{$\ell$}}\\
\bigskip
{\LARGE \tab{$\downarrow$}}\\
\bigskip
This must be a function that takes an integer and multiplies it by n.\\
\vspace*{0.8cm}
\emph{\begin{large}
Type signature of map?\\
$\succ$ map :: (x $\rightarrow$ y) $\rightarrow$ [x] $\rightarrow$ [y]\\
\bigskip
Type signature of x ?\\
$\succ$ Integer $\rightarrow$ Integer $\rightarrow$ Integer\\
\end{large}}
\bigskip
\vspace*{1cm}
{\large \textbf{\underline{Exercise 3.59} in SICP}}\\
\vspace*{0.5cm}
{\normalsize \textbf{\underline{Infinite series}}}\\
\vspace*{0.5cm}
Infinite series  \emph{{\large S}= $ {a}_{0} + {a}_{1}x + {a}_{2}{x}^{2} + {a}_{3}{x}^{3} +\ldots $} treat an infinite series as an infinite list \emph{[a0,a1...]}\\
\bigskip
\hspace{0.9cm}\textbf{Consider} {\large $\int_{y}^{x} S dx$}. Can we write an \underline{integrate} function?\\
\bigskip
\hspace{0.9cm}\textbf{Contract}
[Number] $\rightarrow$ [Number]\\
\begin{large}
$$\int_{0}^{x} {a}_{0} + {a}_{1}x + {a}_{2}{x}^{2} + {a}_{3}{x}^{3} +\ldots dx $$ \\
$$= {a}_{0}x + {a}_{1}\dfrac{x^{2}}{2} + {a}_{2}\dfrac{x^{3}}{3} + {a}_{3}\dfrac{x^{4}}{4}$$
\end{large} 
\begin{flushleft}
\emph{Integrate s = 1 : zipWith\\
\bigskip
Consider a function such that $\int f(x)dx$ = f(x)\\
\tab funny = integrate funny\\
\bigskip
powers x = 1 : map($\ast$ x) (powers x)}\\
\vspace*{1cm}
\textbf{\underline{ZipWith Signature: }}\\
\bigskip
(a $\rightarrow$ b $\rightarrow$ c) $\rightarrow$ [a] $\rightarrow$ [b] $\rightarrow$ [c]
\end{flushleft}
\bigskip
\subsection*{The most beautiful function}
\begin{flushleft}
(LET n ($\nu$ $< expr >$) $< body >$)\\
\bigskip
{\LARGE \tab{$\downarrow$ desugar }}\\
\bigskip
(($\lambda$ $\nu$ $< body >$) $< expr >$)\\
\bigskip
\textcolor{red}{(LET (fac} ($\lambda$ n (((\underline{IF} (\underline{ZERO?} n)) \underline{ONE}) (\underline{fac} (\underline{PRED} n)))))\\
\bigskip
{\LARGE \tab{$\swarrow\downarrow\searrow$}}\\
\bigskip
\textcolor{red}{(fac SEVEN) all defined in LETS}\\
\bigskip

{\LARGE \tab{$\Downarrow$ desugar}}\\
\bigskip
(($\lambda$ fac (fac SEVEN)) ($\lambda$ n ((IF (ZERO? n) ONE) (\textcolor{red}{fac} (PRED  n))))\\

\vspace*{1cm}
Our factorial function needs to look like this: \\
( $\lambda$ fac ($\lambda$ n ..... fac .....))\\
\vspace*{1cm}
Then we need to make a closure of fac applied to \underline{itself}\\
\vspace*{0.4cm}
(\underline{($\lambda$ f (($\lambda$ x (f (x x))) ($\lambda$ x (f (x x)))))} F)\\
\textcolor{red}{\tab{y}}\\
\bigskip
{\LARGE \tab{$\downarrow$ $\beta$}}\\
\bigskip
(($\lambda$ x (F (x x)) ($\lambda$ x (F (x x))))\\
\bigskip
{\LARGE \tab{$\downarrow$ $\beta$}}\\
\bigskip
(F (($\lambda$ x (F (x x)) ($\lambda$ x (F (x x)))) $\equiv$ F(YF)\\
\vspace*{1.5cm}
The \underline{Y combinator}  
\pagebreak

\textcolor{blue}{\textbf{{\huge Week 12 Lecture Notes}}}
\section*{\underline {Haskell}}
\begin{flushleft}
\begin{itemize}
\item has typing
\item and type inference
\end{itemize}
\bigskip
What is the right hand side of cosSeries?\\
It is a list comprehension.\\
\bigskip
{\large Aim is} $\rightarrow$ $ 1 - \dfrac{x^{2}}{2!} + \dfrac{x^{4}}{4!} + \dfrac{x^{6}}{6!}$ .....\\
$$\dfrac{d}{dx}f(x) = f(x)$$
$$\int_{0}^{x} f(x).dx = f(x)$$ \\
$$\int_{0}^{x} sinx dx = 1 - cosx$$\\
$$\int_{0}^{x} cosx = sinx$$
\bigskip
\begin{verbatim}
-- x of f(x) and x which is range on integral (0..x) is the same 
-- f(x) and dx is the same
-- integral x (0..x) and the result f(x) is the same


facs = 1 : zipWith (*) facs [2,3..]
fibs = 1 : 1 : zipWith (+) fibs (tail fibs)
\end{verbatim}
\end{flushleft}
\pagebreak
\section*{\underline{Eratosthenes Sieve}}
\begin{verbatim}
isPrime :: [Integer] Integer ---> Bool
isPrime lp n =
              | (head l p) ^ 2 > n = True
              | mod n (head l p) == 0 False
              | isPrime (tail l p) n
\end{verbatim}
\vspace*{0.5cm}

\section*{\underline{Type Signature}}
\begin{flushleft}
\begin{alltt}
filter :: (a \(\rightarrow\) Bool) \(\rightarrow\) [a] \(\rightarrow\) [a] \\
isPrime :: [Integer] \(\rightarrow\) Integer \(\rightarrow\) Bool\\
map :: (a \(\rightarrow\) b) \(\rightarrow\) [a] \(\rightarrow\) [b] \\
foldr :: (a \(\rightarrow\) b \(\rightarrow\) b) \(\rightarrow\) b \(\rightarrow\) [a] \(\rightarrow\) b \\
sum = foldr (+) \(\emptyset\)
product = foldr (*) 1 \\
Leveraging the Haskell type system - polynomial arithmetic
\end{alltt}
\bigskip
$ ({a}_{0} + {a}_{1}x + {a}_{2}{x}^{2} + \ldots + {a}_{n}{x}^{n})$ $\leftarrow$ polynomial in l variable represent as a list $[{a}_{0}, {a}_{1} \ldots {a}_{n}]$\\
\bigskip
$[{a}_{0}, {a}_{1} \ldots {a}_{n}]$ + $[{b}_{0}, {b}_{1} \ldots {b}_{m}]$ = $[{a}_{0} + {b}_{0}, {a}_{1} + {b}_{1} \ldots]$\\
\bigskip
$[{a}_{0}, {a}_{1} \ldots {a}_{n}]$ * $[{b}_{0}, {b}_{1} \ldots {b}_{m}]$ = $[{a}_{0} * {b}_{0}, {a}_{1} * {b}_{0} + {a}_{0} * {b}_{1} \ldots {a}_{m} * {b}_{n}]$\\
\pagebreak
\textcolor{blue}{\textbf{{\huge Week 13 Lecture Notes}}}
\section*{Memory Management}
\begin{flushleft}
\verb|int[] a = new int[n]; // write n instead of value such 100 | 
\end{flushleft}
\bigskip
If we have \underline{manual} memory management:
\begin{flushleft}
In C/C++ we have \underline{malloc} and \underline{free}.\\
\bigskip
\begin{itemize}
\item[(a)] We can have a memory leak. (forget to free)
\item[(b)] We can have a catastrophe. (free memory still in use)
\end{itemize}
\bigskip
Even this is not cheap.\\
Especially if we think about \underline{fragmentation}.\\
\vspace*{1cm}
\framebox(200,20){\huge $|$ free $|$      $|$ free $|$      $|$ free $|$}\\
\vspace*{1cm}
Manual is not a good idea. - Leads to problems in practice.\\
\vspace*{0.1cm}
We want an automatic systems.\\
\vspace*{0.1cm}
Needs to detect when allocated memory can be handed back.
\bigskip
\begin{itemize}
  \item We want a system that is \underline{sound}. (does not hand back free memory)
  \item We want a system that is \underline{complete}. (hands back all free memory)
\end{itemize}
\end{flushleft}
\vspace*{1cm}
\subsection*{\underline{Solution 1} (PHP, Swift)}
\subsubsection*{\underline{Reference Counts}}
 \begin{flushleft}
Increase for each new reference to a piece of memory,\\
\vspace*{0.1cm}
Because when a reference is deleted or overwritten.\\
\vspace*{0.1cm}
When the count drops to \underline{zero} hand back the memory.
\bigskip
  \begin{verbatim}
   class X {
      .. X a = this;    <-- Circular Reference // cause a memory leak
   }
\end{verbatim}
\end{flushleft}
\subsection*{\underline{Solution 2} (Java, C\#, Racket, Haskell)}
\subsubsection*{\underline{Garbage Collection}}
\begin{flushleft}
\vspace*{0.4cm}
\framebox(100,100){\huge $\blacksquare$ $\blacksquare$}\\
\vspace*{0.8cm}
Find the memory in use starting from base references,\\
\vspace*{0.1cm}
Mark the memory in use,\\
\vspace*{0.1cm}
Free all the rest.\\
\vspace*{1cm}

\textbf{\underline {Cheney} (1970)}\\
\vspace*{0.4cm}
Divide memory into two,\\
\vspace*{0.5cm}
\framebox(100,100){\huge $\blacksquare$ $\blacksquare$ $\blacksquare$  $\rightarrow$}\framebox(100,100){\huge \boxed{}\boxed{}\boxed{}}\\
\hspace*{1.5cm} A\hspace*{3cm} B\\
\vspace*{0.8cm}
Use A,\\
\vspace*{0.1cm}
When A is "exhausted"\\
\vspace*{0.1cm}
Copy referenced memory from A to B\\
\vspace*{0.1cm}
Mark the old copies with ''broken hearts"\\
\vspace*{0.8cm}
\textbf{Exploit different kinds of memory allocation}\\
\begin{itemize}
 \item[-] Short term allocation
 \item[-]Long term allocation
\end{itemize}

\subsection*{\underline{Generational Garbage Collection}}
\vspace*{0.4cm}
\begin{verbatim}
 (define (substitute [to-sub : lambda] [for : symbol] [in : lambda]) : lambda
      (type-case lambda in
           [lambda-id (id) (if (symbol=? for id) to-sub in)]
           [lambda-apply (lhs rhs) (lambda-apply (substitute to-sub for lhs)
                                                 (substitute to-sub for rhs))]
           [lambda-def (bound body) (if (symbol=? for bound) in
                                    (if (or (not (set-member for (free-identifiers body)))
                                            (not (set-member bound (free-identifiers to-sub)))
                                    (lambda-def bound (substitute to-sub in body)))))]
\end{verbatim}
\noindent\makebox[\linewidth]{\rule{\paperwidth}{0.4pt}}\\
\vspace*{1.5cm}
\textcolor{blue}{\textbf{{\huge Week 14 Lecture Notes}}}

\section*{Last Lecture}
\begin{flushleft}
\vspace*{0.3cm}
(LET ((MUL ($\lambda$ m ($\lambda$ m ($\lambda$ f ($\lambda$ x (m (n x)))))))) body)\\
\vspace*{0.2cm}
(LET \underline{((id1 bv1) (id2 bv2)} ..) $<$ body $>$))\\
\vspace*{0.2cm}
(LET (id bv) $<$ body $>$) $\rightarrow$ (($\Lambda$ id $<$ body $>$) bv)\\
\vspace*{0.2cm}
(LET (id1 bv1) (LET (id2 bv2) (LET..) $<$ body $>$)) // this is fold\\
\vspace*{0.8cm}
\pagebreak
\textbf{Quiz last question result}\\
\vspace*{0.2cm}
(LET ((a (b x )) (b c)) ($\lambda$ b (c a)))\\
\hspace*{0.5cm}$\downarrow$\hspace*{2cm}$\searrow$\\
(LET ((\textcolor{red}{a} (\textcolor{blue}{b} x))) (LET ((b c)) ($\lambda$ b (c a))))))\\
\vspace*{0.2cm}
(($\lambda$ \textcolor{red}{a} (($\lambda$ b ($\lambda$ b (c a)) c))) (\textcolor{blue}{b} x)))\\
\vspace*{1.5cm}
To get a \underline{practical} lambda calculus interpreter we need to always do the \underline{leftmost} $\beta$ transformation.\\
\bigskip
\textbf{Example} Y combinator will loop if you don't always do the leftmost\\
\vspace*{0.3cm}
Consider ($\Lambda_{1}$ $\Lambda_{2}$) // $\Lambda_{1}$ is not a def\\
\vspace*{0.3cm}
So we transform $\Lambda_{1}$ - if after \underline{any} transformation $\Lambda_{1}$ becomes a def we need to apply it immediately.
\end{flushleft}

\vspace*{1cm}
\subsection*{\underline{Mutation + sequence}}
\begin{lstlisting}[mathescape]
  a=b;       c=a;
	 $\not\equiv$          $\Rightarrow$ Sequence matters (purelly functional data structures Chris Okosaki)
  c=a;       a=b;
                      Memory is a key value store
                      
 (begin  a1
         a2      $\Rightarrow$ Make every value a pair Mem x value (a4 (a3 (a2 (a1 x))))
         a3
         a4 ) $\rightarrow$ mutation\\
\end{lstlisting}
\vspace*{0.7cm}
 \begin{flushleft}
A general solver for polynomial equations of this form $a_{0}$ + $a_{1}x$ + $a_{2}x^{2}$ + .. + $a_{n}x^{n}$ = 0 represent polynomial as array of number.\\
\vspace*{0.3cm}
\hspace*{1.5cm}(1 + 2x + $3x^{2}$) (4 + 2x) = 4 + 10x + $16x^{2}$ + $6x^{3}$\\
\vspace*{0.3cm}
evaluate $a_{0}$ + $a_{1}x$ + .. + $a_{n}x^{n}$ = $a_{0}$ + x($a_{1}$ + x($a_{2}$ .. n)))))\\
\vspace*{0.3cm}
\hspace*{1.5cm} 1 + 2x + $3x^{2}$ + $4x^{3}$ , x=2 $\rightarrow$ 49\\
Make a polynomial where roots are $r_{0}$ $r_{1}$ $r_{2} ...$\\
\vspace*{0.4cm}
\hspace*{0.5cm} (x - $r_{0}$) (x - $r_{1}$) (x - $r_{2}$) ... (x - $r_{n}$)\\
\vspace*{0.2cm}
\hspace*{0.5cm} (-$r_{0}$ + x) (-$r_{1}$ + x) (-$r_{2}$ + x) ... (-$r_{n}$ + x)\\
\vspace*{0.6cm}
\hspace*{0.5cm} (x - 1) (x - 2) (x - 3) (x - 4)\\
\vspace*{0.2cm}
\hspace*{0.5cm} $x^{4}$ - $10x^{3}$ + $35x^{2}$ - 50  + 24\\
\vspace*{1cm}
\underline{\textbf{Need to differentiate a polynomial}}\\
\vspace*{0.4cm}
\hspace*{0.3cm} $a_{0}$ + $a_{1}x$ + $a_{2}x^2$ ... $\rightarrow$  $a_{1}$ + $2a_{2}x$ + $3a_{3}x^2$ + $na_{n}x^{n-1}$
  
\end{flushleft}
\end{flushleft}
\end{flushleft}
\end{flushleft}
\end{flushleft}
\end{flushleft}

\end{document}