\documentclass{article}
\usepackage[margin=2cm,bottom=2cm]{geometry}
\usepackage{comment}
\usepackage[utf8]{inputenc}
\usepackage{graphicx}
\usepackage{mathtools}
\usepackage[normalem]{ulem}
\usepackage{setspace}
\usepackage{MnSymbol}
\usepackage{soul}
\usepackage{amsmath}

\begin{document}
\title{COMP/CMPE 314 - Principles of Programming Languages - Notes}
\author{Chris Stephenson, Istanbul Bilgi University, Department of Mathematics, and course students}
\maketitle

\begin{flushleft}
 Use map to write Haskell function that multiplies every element of a possibly infinite list of integers by n.\\
 \begin{verbatim}
  mulList :: Integer -> [Integer] -> [Integer]
  mulList n l = map ((*) n) l
 \end{verbatim}
The signature of map?\\
\begin{verbatim}
 map :: (x -> y) -> [x] -> [y]
\end{verbatim}
Type signature of *?
\begin{verbatim}
 Integer -> Integer -> Integer
\end{verbatim}
\end{flushleft}
\section*{Exercise 3.59 in SICP}
\begin{flushleft}
 \underline{Infinite series}\\
 \verb|S = |  $a_0 + a_1x + a_2x^2 + a_3x^3 + ... $\\
 treat an infinite series as an infinite list $[a_0, a_1]$\\
 \bigskip
 Consider $\int_{0}^{x} Sdx$ can we write an \underline{integrate} function?\\
 \bigskip
 Contract \verb|[Number] -> [Number]|\\
 $$\int_{0}^{x} (a_0 + a_1x + a_2x^2 + a_3x^3 + ...) dx = a_0x + \frac{a_1x^2}{2} + \frac{a_2x^3}{3} + ...$$\\
 \begin{verbatim}
  integrate s = 1 : zipWith (/) S [1,2..]
 \end{verbatim}
 \underline{zipWith signature}\\
 \verb|(a -> b -> c) -> [a] -> [b] -> [c]|\\
 \bigskip
 Consider a function such that $\int f(x)dx = f(x)$\\
 \begin{verbatim}
  funny = integrate funny
  
  powers x = 1 : map(* x) (powers x)
 \end{verbatim}
\end{flushleft}
\pagebreak
\section*{The most beautiful function}
\begin{flushleft}
 \begin{verbatim}
  (LET ((x (lambda x x)) (succ (lambda n (lambda f (lambda x (f ((n f) x))))))
         <body>)
         
         
  (LETn (v <expr>) <body>)  ==>  ((lambda v <body>) <expr>)
 \end{verbatim}
 we want to write factorial\\
 \begin{verbatim}
  (LET (fac (lambda n (((IF (ZERO ? n)) ONE) (fac (PRED n))))
 \end{verbatim}
 all defined in LETs desugar
 \begin{verbatim}
  ((lambda fac (fac SEVEN)) (lambda n ((IF (ZERO ? N) ONE) (fac (PRED n))))
 \end{verbatim}
 Our factorial function needs to look like this
 \begin{verbatim}
  (lambda fac (lambda n .. fac ..))
 \end{verbatim}
 Then we need to make a closure of fac applied to \underline{itself.}
 \begin{verbatim}
  ((lambda f ((lambda x (f (x x))) (lambda x (f (x x))))) F)
 \end{verbatim}
 $\Big\downarrow{\beta}$\\
 \bigskip
 \verb|((lambda x (F (x x)) (lambda x (F (x x))))| $\equiv$ \verb|(YF)|\\
 \bigskip
 $\Big\downarrow{\beta}$\\
 \bigskip
 \verb|(F ((lambda x (F (x x)) (lambda x (F (x x))))| $\equiv$ \verb|F(YF)|\\
 \bigskip
 \underline{The Y combinator}
\end{flushleft}
\end{document}
