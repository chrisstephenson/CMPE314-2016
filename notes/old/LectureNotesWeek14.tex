\documentclass{article}
\usepackage[margin=2cm,bottom=2cm]{geometry}
\usepackage{comment}
\usepackage[utf8]{inputenc}
\usepackage{graphicx}
\usepackage{mathtools}
\usepackage[normalem]{ulem}
\usepackage{setspace}
\usepackage{MnSymbol}
\usepackage{soul}
\usepackage{amsmath}

\begin{document}
\title{COMP/CMPE 314 - Principles of Programming Languages - Notes}
\author{Chris Stephenson, Istanbul Bilgi University, Department of Mathematics, and course students}
\maketitle

\section*{Week14}
\begin{flushleft}
 \begin{verbatim}
  (LET ((MUL (lambda m (lambda m (lambda f (lambda x (m (n x)))))))) body)
  
  (LET ((id1 bv1) (id2 bv2) ...) <body>)
  
  (LET (id bv) <body>) --> ((lambda id <body>) bv)
  
  (LET (id1 bv1) (LET (id2 bv2) (LET ...) <body>)
 \end{verbatim}
 this is a \underline{fold.}\\
 \bigskip
 \underline{\textbf{Quiz question answer:}}\\
 \begin{verbatim}
  (LET ((a (b x)) (b c)) (lambda b (c a)))
  
  (LET ((a (b x))) (LET ((b c)) (lambda b (c a))))
  
  ((lambda a ((lambda b (lambda b (c a))) c))) (b x)))
 \end{verbatim}
 \bigskip
 To get a \underline{practical} lambda calculus interpreter we need to always do the \underline{leftmost} $\beta$ transformation.\\
 \underline{Example:} Y combinator will loop if you don't always do the leftmost.\\
 \bigskip
 Consider ($\Lambda_1$ $\Lambda_2$) $\Lambda_1$ is not a def \\
 So we transform $\Lambda_1$ - if after \underline{any} transformation $\Lambda_1$ becomes a def we need to apply it immediately.
\end{flushleft}

\subsection*{Mutation + Sequence}
\begin{flushleft}
 \begin{verbatim}
  a = b;                           c = a;
               not equal to        
  c = a;                           a = b;
 \end{verbatim}
 \underline{Sequence matters!}\\
 \bigskip
 in Racket;\\
 \begin{verbatim}
  (begin
      a1
      a2
      a3
      a4)
 \end{verbatim}
 memory is a \underline{key-value} store\\
 make every value a pair \underline{mem x value}\\
 \verb|(a4 (a3 (a2 (a1 x))))|
 \bigskip
 A general solver for polynomial equations of this form\\
 $a_0 + a_1x + a_2x^2 + ... + a_nx^n = 0$\\
 Present polynomial as array of numbers\\
 $(1 + 2x + 3x^2) (4 + 2x) = 4 + 10x + 16x^2 + 6x^3$\\
 evaluate $a_0 + a_1x + ... + a_nx^n = a_0 + x(a_1 + x(a_2 ... a_n)))))$\\
 $1 + 2x + 3x^2 + 4x^3$  , x=2 $\rightarrow$ 49\\
 Make the polynomial whose roots are $r_0, r_1, r_2, ...$\\
 $(x - r_0) (x - r_1) (x - r_2) ... (x - r_n)$\\
 $(-r_0 + x) (-r_1 + x) (-r_2 + x) ... (-r_n + x)$\\
 $(x - 1) (x - 2) (x - 3) (x - 4)$\\
 $x^4 - 10x^3 + 35x^2 - 50 + 24$\\
 \underline{Need to differentiate a polynomial}\\
 $a_0 + a_1x + a_2x^2 ... $ $\rightarrow$ $a_1 + 2a_2x + 3a_3x^2 + ... + na_nx^{n-1}$
\end{flushleft}
\end{document}
