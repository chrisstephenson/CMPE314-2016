\documentclass{article}
\usepackage[margin=2cm,bottom=2cm]{geometry}
\usepackage{hyperref}
\usepackage{comment}
\usepackage[utf8]{inputenc}
\usepackage{graphicx}
\usepackage{mathtools}
\usepackage[normalem]{ulem}
\usepackage{MnSymbol}
\usepackage{tikz}
\def\checkmark{\tikz\fill[scale=0.4](0,.35) -- (.25,0) -- (1,.7) -- (.25,.15) -- cycle;} 

\newcommand\tab[1][1cm]{\hspace*{#1}}
\DeclareMathSizes{10}{10}{10}{10}

\begin{document}
\title{COMP/CMPE 314 - Principles of Programming Languages - Notes}
\author{Chris Stephenson, Istanbul Bilgi University, Department of Mathematics, and course students}
\maketitle
 
\begin{flushleft}
\section*{A program that looks the same but evaluates to two different answers should worry us!}
The following function/method is both works on Java and C, but when we compile it gives different results.
\begin{verbatim}
int funny (int a, int b){
  return a * 2 + b;
}

int haha () {
  int z = 2;
  return funny (z++, z++);
}
\end{verbatim}
This method/function is legit for both languages.\\
\bigskip
In C, we can compile it with one easy step. For this, we need to add C header files and main function.
\begin{verbatim}
#include<stdio.h>

int funny (int a, int b){
  return a * 2 + b;
}

int haha () {
  int z = 2;
  return funny (z++, z++);
}

int main(int argc, char** argv) {
  printf("\n\nThe answer in C is %d\n\n", haha());
  return 0;
}
\end{verbatim}
The output of the program is:\\
\verb|The answer in C is 8|\\
\bigskip
In Java we need to change the methods a little bit. We should add static at the beginning of the methods in order to compile.\\
We can do this without changing the main file via terminal.\\
\begin{verbatim}
 cat funnypre.java | cat - funnyhaha | sed -e 's/^int/static int/g' | cat - funnypost.java
\end{verbatim}
(The files are in the course github repository.)\\
\pagebreak
Output of the code is:
\begin{verbatim}
class Funny{

static int funny (int a, int b){
  return a * 2 + b;
}

static int haha () {
  int z = 2;
  return funny (z++, z++);
} 

public static void main(String[] args){
  System.out.println("\n\nThe answer in Java is " + haha() + "\n\n");
}

}
\end{verbatim}
The output of the program is:\\
\verb|The answer in Java is 7|\\
\bigskip
We can write and compile codes without using Eclipse.\\
Integrated Development Environment (IDE) $ \rightarrow $ Eclipse\\
Unintegrated Development Environment $ \rightarrow $ Terminal\\
\bigskip
For repetitive jobs we should write bash scripts.\\
Example for C:
\begin{verbatim}
cat funnypre.c funnyhaha funnypost.c > funny.c
gcc funny.c -o funny
./funny
\end{verbatim}

\centerline{\noindent\rule{18cm}{0.4pt}}
\bigskip
\textbf{z++}\\
\bigskip
The \verb|++| post operator produces a value the same as the value of its operand. But it changes the subsequent value of the operand as a \underline{side effect.}\\ 
In a program using a for loop to process an array we have (maybe) unnecessary state.\\
Look at Hadoop - Map reduce $ \leftarrow $ must be stateless (big data)\\
\bigskip
\centerline{text $\xrightarrow{parse}$ first representation $\xrightarrow{desugar}$ second representation $\xrightarrow{eval}$ evaluation}
\centerline{"desugaring" = "macro processing"}
\bigskip
For example in Racket, cond is desugared into a series of nested if statements.\\
\bigskip
What does our program evaluation look like?
\begin{verbatim}
 (eval (desugar (parse '(* (-3) (+ 7 8)))))
\end{verbatim}

\centerline{\noindent\rule{18cm}{0.4pt}}
\pagebreak
\textbf{Rules:}\\
$\lambda \rightarrow \mathit{v}$  identifier\\
$\lambda \rightarrow ( \lambda \mathit{v} \Lambda )$ anonymous function definition\\
$\lambda \rightarrow ( \Lambda \Lambda )$ function application \\

\bigskip
$( \mathit{a} $ $ ( \mathit{b}$ $ \mathit{c} ))$ \checkmark\\
$\mathit{a}$ \checkmark\\
$(\mathit{a})$ x \\
$(\mathit{a}$ $\mathit{b})$ \checkmark\\

\centerline{\noindent\rule{18cm}{0.4pt}}
\bigskip
$(\lambda$ $\mathit{x}$ $\mathit{x}) \rightarrow$ identity function$$$$
\begin{verbatim}
int funny(int y){
  return y;
}
\end{verbatim}
$((\lambda$ $\mathit{x}$ $\mathit{x})$ $(\mathit{a}$ $\mathit{b}))$\\
In here x is bound, a and b are free\\
\bigskip
A "complete" program will have no "free" identifiers.\\
A part of that program may have free identifiers.\\

\begin{itemize}
 \item If $\mathit{M}$ is an identifier, $\mathit{x}$. $\mathit{FI(M) = {X}}$
 \item If $\mathit{M}$ is an application $(\mathit{M1}$ $ \mathit{M2})$. $\mathit{FI(M) = FI(M1) \cup FI(M2)}$
 \item If $\mathit{M}$ is a definition $(\lambda$ $\mathit{x}$ $\mathit{M1})$. $\mathit{FI(M1) - }$ \{$\mathit{x}$\}
\end{itemize} 
\bigskip
$\mathit{FI[(\lambda x \space{} (\lambda \space{} y \space{} x ))] = \{\}}$\\
$\mathit{FI[(\lambda y x)] = \{x\}}$
\centerline{\noindent\rule{18cm}{0.4pt}}
\bigskip
\textbf{If statement}\\
True:\\
$\mathit{T = (\lambda x (\lambda y x))}$\\
$\mathit{(T \space{} a) \xrightarrow{\beta} (\lambda \space{} y \space{} a)}$\\
$\mathit{((T \space{} a) \space{} b) \xrightarrow{\beta} a}$\\
\bigskip
False:\\
$\mathit{F = (\lambda \space{} x \space{} (\lambda \space{} y \space{} y))}$\\
$\mathit{((F a) b) \xrightarrow{\beta} b}$

\end{flushleft}

 
\end{document}
