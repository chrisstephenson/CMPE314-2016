\documentclass{article}
\usepackage[margin=2cm,bottom=2cm]{geometry}
\usepackage{hyperref}
\usepackage{comment}
\usepackage[utf8]{inputenc}
\usepackage{graphicx}
\usepackage{mathtools}
\usepackage[normalem]{ulem}
\usepackage{color}
\usepackage{setspace}
\usepackage{MnSymbol}
\usepackage[T1]{fontenc}

\newcommand\tab[1][1cm]{\hspace*{#1}}
\DeclareMathSizes{10}{10}{10}{10}

\begin{document}
\title{COMP/CMPE 314 - Principles of Programming Languages - Notes}
\author{Chris Stephenson, Istanbul Bilgi University, Department of Mathematics, and course students}
\maketitle

\section*{Lies and Equality (In Java and Other Languages)}
In Java;
\begin{itemize}
 \item "\verb|=|" does not mean "equals" (its assignment)
 \item "\verb|&|" does not mean "and" (its bitwise and)
 \item "|" does not mean "or" (its bitwise or)
 \item "\verb|!|" does not mean "surprise"
 \item "\verb|int|" does not mean "integer"
 \item "\verb|Integer|" does not mean "integer"
 \item "\verb|BigInteger|" does not mean "BigInteger"
\end{itemize}
\bigskip
\begin{itemize}
 \item We use "\verb|==|" for "equals"
 \item We use "\verb|&&|" for "logical and"
 \item We use "||" for "logical or"
\end{itemize}
\begin{flushleft}
 This things are left from C which was invented as high level assembler, not high level language.\\
 \verb|int| \underline{means} a 32 bit two's complement quantity (now most compilers are 64 bit)
\end{flushleft}
\bigskip
\begin{flushleft}
 If I declare \verb|MyClass a;|\\
 \verb|a| is not necessarily an object of class \verb|MyClass|.
\end{flushleft}
\bigskip
\section*{Equality}
\subsection*{What does == mean?}
\begin{verbatim}
  class Equality {

  public static void main (String[] args) {
    MyInteger a = new MyInteger(42);
    MyInteger b = new MyInteger(42);
    MyInteger c = new MyInteger(43);
    MyInteger d = c;
    String s1 = "Chris";
    String s2 = "Chris";
    System.out.println("\n a = " + a + ",b = " + b + "\n");
    System.out.println("\n a == b is " + (a == b) + "\n");
    System.out.println("\n s1 == s2 is " + (s1 == s2) + "\n");
    System.out.println("\nd = " + d + "\n");
    System.out.println("\n c == d is " + (c == d) + "\n");
    c.setV(44);
    System.out.println("\n c = " + c +  "\n");
  }
}
\end{verbatim}
\bigskip
\begin{verbatim}
class MyInteger {
  int value;
  
  MyInteger (int v) {
    value = v;
  }

  public String toString () {
    return "" + value;
  }

  public void setV (int v) {
    value = v;
  }
}
\end{verbatim}
\bigskip
Output:
\begin{verbatim}
 a = 42,b = 42


 a == b is false


 s1 == s2 is true


d = 43


 c == d is true


 c = 44
\end{verbatim}

\begin{flushleft}
 It does \underline{not} mean "equal in value" (necessarity)\\
 \verb|setV| method makes our program mutable.\\
 Java is broken. \verb|NullPointerException| is most likely seen error and it causes programs to crash.\\
 The best way to solve this problem is to make our programs \textbf{immutable}.
\end{flushleft}
\begin{verbatim}
 String a = "Chris";
 String b = "Chris";
\end{verbatim}
\begin{flushleft}
 In Java Strings are immutable!
\end{flushleft}
\url{http://javapractices.com}\\
\bigskip
\centerline{\noindent\rule{18cm}{0.4pt}}
\bigskip
\begin{flushleft}
 Our program;\\
 \begin{verbatim}
  '(((fundef factorial n (if2 n 1 (* n (factorial (- n 1))))) (factorial 3))
 \end{verbatim}
 parse + desugar
 \begin{verbatim}
  (appC 'factorial (numC 3))
 
  (ifzC (idC 'n) (numc 1) (mulC (idC 'n) (appC 'factorial (subC (idC 'n) (numC1)))
  
  (idC 'n)'s will be changed with (numC 3)
 \end{verbatim}
 \end{flushleft}
\bigskip
\centerline{\noindent\rule{18cm}{0.4pt}}
\section*{$\alpha$ Transformation}
($\lambda$ $\mathit{x}$ $\mathit{M}$) $\equiv$ ($\lambda$ $\mathit{y}$ $\mathit{M}$ $\mathit{[}$ $\mathit{x}$:=$\mathit{y}$ $\mathit{]}$)

\section*{$\eta$ Transformation (Optional)}
($\lambda$ $\mathit{x}$ ($\mathit{f}$ $\mathit{x}$)) $\xrightarrow{\eta}$ $\mathit{f}$

\section*{$\beta$ Transformation}
(($\lambda$ $\mathit{x}$ $\mathit{M}$) $\mathit{N}$) $\xrightarrow{\beta}$ $\mathit{M}$ $\mathit{[}$ $\mathit{x}$ := $\mathit{N}$ $\mathit{]}$

\end{document}

