\documentclass{article}
\usepackage[margin=2cm,bottom=2cm]{geometry}
\usepackage{hyperref}
\usepackage{comment}
\usepackage[utf8]{inputenc}
\usepackage{graphicx}
\usepackage{mathtools}
\usepackage[normalem]{ulem}
\usepackage{setspace}
\usepackage{MnSymbol}
\usepackage{soul}

\newcommand\tab[1][1cm]{\hspace*{#1}}
\DeclareMathSizes{10}{10}{10}{10}

\begin{document}
\title{COMP/CMPE 314 - Principles of Programming Languages - Notes}
\author{Chris Stephenson, Istanbul Bilgi University, Department of Mathematics, and course students}
\maketitle

\section*{Evaluation "Strategies"}
\underline{All about function application} - which is the central concept in programming \textbf{IMHO} an oxymoron
\begin{flushleft}
 We looked at \underline{sequence}. (2 weeks ago)
 \begin{itemize}
  \item Strict vs. non-strict
  \item Eager vs. lazy
  \item Left to right vs. right to left
 \end{itemize}
\underline{What} is passed when we apply a function to actual parameter.\\
Example (in Java)\\
\begin{verbatim}
 {
  a = 2;
  myFunc(a);
 }
 
 void myFunc(int a){
  ...
  a = a + 1;
 }
\end{verbatim}
What is the value of \verb|a|?\\
\verb|System.out.println("a is: " + a);| $\rightarrow$ The answer is 2.
\end{flushleft}
\subsection*{Why?}
\begin{flushleft}
 Because Java/C/C\#/Ruby/Python/C++ are, \underline{call by value}\\
 For simple types, most common programming languages are call by value.\\
 The alternative is \underline{call by preference} (PHP \underline{can} do this)
\end{flushleft}
\section*{But what about complex types?}
\underline{Example:} array, structure, object\\
In Java:
\begin{verbatim}
 {
  ...
  a[k] = 2;
  myFunc(a);
  System.out.println("a[k] is " + a");
 }
 
 
 void myFunc(int[] a){
  ...
  a[l] = a[l] + 1;
  ...
 }
\end{verbatim}
\begin{flushleft}
Call by reference for arrays/Structures/Strings/Objects\\
(Except C - C structures are naturally call by value)\\
Java Strings can be considered call by value. (Because Java Strings are immutable)
\end{flushleft}

\section*{SUPER ACCELERATED CMPE313 (SICP - The Purple Book)}
\begin{flushleft}
 What is a list? A list of \st{numbers} x?
 \begin{itemize}
  \item Empty
  \item A pair of a \st{number} x and a list of \st{number} x.
 \end{itemize}
 \begin{verbatim}
  ;; add-up the numbers in a list
  (define (sum-lon lon)
    (cond
      ((empty? lon) 0)
      (else (+ (first lon) (sum-lon (rest lon))))))
      

  (define (mul-lon lon)
    (cond
      ((empty? lon) 1)
      (else (* (first lon) (mul-lon (rest-lon))))))
      
      
  (define (fold combiner null-value lox)
    (cond
      ((empty? lox) null-value)
      (else (combiner (first lox) (fold combiner null-value (rest lox))))))
 \end{verbatim}
 \textbf{fold} is a powerful \underline{abstraction}.\\
 \bigskip
 I want to evaluate a polynomial.\\
 $ {a}_{0} + {a}_{1}x + {a}_{2}{x}^{2} + {a}_{3}{x}^{3} +....+ {a}_{n}{x}^{n} $ \\
 Data definition a polynomial number $\rightarrow$ number\\
 \bigskip
 Horner's rule;\\
 ${a}_{0} + x({a}_{1} + x({a}_{2} + x({a}_{3}...{a}_{n}))))$\\
 \bigskip
 To evaluate a polynomial\\
 \begin{verbatim}
  (define (eval-poly p x)
    (fold (lambda (l r) (+ l (+ l (* x r)) non-empty p))))
 \end{verbatim}
 map
 \begin{verbatim}
  (define (map f lox) 
    (fold (lambda (l r) (cons (f l) (cons (f l) r)) empty lox)))
 \end{verbatim}
\bigskip
 filter
\begin{verbatim}
 (define (filter p? lox)
    (fold (lambda (l r) (if (p? l) (cons l r) empty lox))))
\end{verbatim}
Suppose I am doing a \underline{map} to add 2 to every item in a list:
\begin{verbatim}
 (define (add2 lon)
    (map (lambda (x) (+ x 2)) lon))
    
 (define (mul 7 lon)
    (map (lambda (x) (* x 7)) lon))
    
 (define (curry f x)
    (lambda (y) (f x y)))
\end{verbatim}
Generalizing:
\begin{verbatim}
 (define (arith-map operator constant)
    (map (curry operator constant) lon))
\end{verbatim}
\end{flushleft}
\section*{The $\lambda$-calculus}
\subsubsection*{Summary:}
\begin{flushleft}
$\Lambda\rightarrow\upsilon$ (identifier) \\ 
$\Lambda\rightarrow(\lambda$ $\upsilon$ $\Lambda)$ (definition) \\
$\Lambda\rightarrow(\Lambda$ $\Lambda)$ (application) \\
\bigskip
\underline{$\alpha$ rewriting rule:}\\
($\lambda$ $\upsilon$ $\Lambda$) $\equiv$ ($\lambda$ $\mathit{u}$ $\Lambda$[$\upsilon$:=$\mathit{u}$]) (plagiarizing student rule)\\
\bigskip
\underline{$\eta$ rule (optional):}\\
($\lambda$ $\upsilon$ (f $\upsilon$)) $\equiv$ f\\
\bigskip
\underline{$\beta$ rule:}\\
(($\lambda$ $\upsilon$ ${\Lambda}_{1}$) ${\Lambda}_{2}$ $\xrightarrow{\beta}$ ${\Lambda}_{1}$ [$\upsilon$:=${\Lambda}_{2}$]\\
\bigskip
defining substitution is more complicated\\
$\Lambda$[$\upsilon$:=${\Lambda}_{0}$] - see the slides \\
\begin{itemize}
 \item[(a)] $\Lambda$ = $\mathit{u}$
 \item[] if $\mathit{u}$ = $\upsilon$ $\rightarrow$ ${\Lambda}_{0}$ 
 \item[] else $\Lambda$
 \item[(b)] $\Lambda$ = (${\Lambda}_{1}$ ${\Lambda}_{2}$)
 \item[] (${\Lambda}_{1}$[$\upsilon$=${\Lambda}_{0}$] ${\Lambda}_{2}$[$\upsilon$=${\Lambda}_{0}$]
 \item[(c)] $\Lambda$ = ($\lambda$ $\mathit{u}$ ${\Lambda}_{1}$)
 \subitem$\mathit{(i)}$ $\mathit{u}$ = $\upsilon$ $\rightarrow$ $\Lambda$
 \subitem$\mathit{(ii)}$ $\mathit{u}$ $\neq\upsilon$ $\rightarrow$ if necessary re-letter before substituting in the body.
\end{itemize}
\end{flushleft}
\pagebreak
\section*{Church Numerals}
\begin{flushleft} 
 0 $\equiv$ ($\lambda$ $\mathit{f}$ ($\lambda$ $\mathit{x}$ $\mathit{x}$))\\
 1 $\equiv$ ($\lambda$ $\mathit{f}$ ($\lambda$ $\mathit{x}$ ($\mathit{f}$ $\mathit{x}$)))\\
 2 $\equiv$ ($\lambda$ $\mathit{f}$ ($\lambda$ $\mathit{x}$ ($\mathit{f}$ ($\mathit{f}$ $\mathit{x}$))))\\
\bigskip
\textbf{Successor:}\\
SUCC=($\lambda$ n ($\lambda$  $\mathit{f}$  ($\lambda$  $\mathit{x}$  ( $\mathit{f}$  ((n  $\mathit{f}$ )  $\mathit{x}$ ))))\\
onSUCC=($\lambda$ n ($\lambda$  $\mathit{f}$  ($\lambda$  $\mathit{x}$  ((n  $\mathit{f}$ ) ( $\mathit{f}$   $\mathit{x}$ ))))\\
\bigskip
(($\lambda$ n ($\lambda$  $\mathit{f}$  ($\lambda$  $\mathit{x}$  ( $\mathit{f}$  ((n  $\mathit{f}$ )  $\mathit{x}$ ))))) $\lambda$  $\mathit{f}$  ($\lambda$  $\mathit{x}$  ( $\mathit{f}$   $\mathit{x}$ ))))\\
\bigskip
$\downarrow \beta$\\
\bigskip
($\lambda$  $\mathit{f}$  ($\lambda$  $\mathit{x}$  ( $\mathit{f}$  (($\lambda$  $\mathit{f}$  ($\lambda$  $\mathit{x}$  ( $\mathit{f}$   $\mathit{x}$ )))  $\mathit{f}$ )  $\mathit{x}$ ))))\\
\bigskip
$\downarrow \beta$\\
\bigskip
($\lambda$  $\mathit{f}$  ($\lambda$  $\mathit{x}$  ( $\mathit{f}$  (($\lambda$  $\mathit{x}$  ( $\mathit{f}$   $\mathit{x}$ )))  $\mathit{x}$ ))))\\
\bigskip
$\downarrow \beta$\\
\bigskip
($\lambda$  $\mathit{f}$  ($\lambda$  $\mathit{x}$  ( $\mathit{f}$  ( $\mathit{f}$   $\mathit{x}$ )))) $\equiv$ 2!
\end{flushleft}

\end{document}
