\documentclass{article}
\usepackage[margin=2cm,bottom=2cm]{geometry}
\usepackage{hyperref}
\usepackage{comment}
\usepackage[utf8]{inputenc}
\usepackage{mathtools}
\usepackage{color}
\usepackage{amsmath}
\usepackage{array}
\usepackage{setspace}
\usepackage{soul}
\usepackage{amssymb}
\usepackage{graphicx}
\usepackage{tcolorbox}

\begin{document}
\title{COMP/CMPE 314 - Principles of Programming Languages - Notes}
\author{Chris Stephenson, Istanbul Bilgi University, Department of Mathematics, and course students}
\date{April,21}
\maketitle

\section{Haskell}
\begin{flushleft}
\underline{(Simon Reyton-Jones)} $\rightarrow$ Glasgow Haskell Compiler new language \underline{F\#}
\begin{itemize}
\item Purely (no assignment, no side effects) Functional-function values are first class value.
\item \underline{Lazy} (example: all lists can be infinite. No special treatment is needed for infinite lists)
\item Sophisticated Type System (That allows generic programming) Variable types.
\item Lots of interesting sugar.
\end{itemize}
\begin{flushleft}
\begin{tcolorbox}
Java is not the solution\\
Java is the problem
\end{tcolorbox}
\bigskip
\end{flushleft}
	
\subsection*{$\beta$ transformation}
\begin{flushleft}
Why did we object to identifier capture?\\
\begin{flushleft}
\doublespacing 
($\lambda$ m (((($\lambda$ x (($\lambda$ y (x y))) m) n))))) y) x)\\ ($\lambda$ m (((($\lambda$ x (($\lambda$ y (x y))) m) n))) y) x)
\end{flushleft}
\textcolor{red}{If we have a identifier capture the result of the B transformations depends on the order they are done in.}
\end{flushleft}
\begin{flushleft}
Is it the case that our definition of B transformation (including the substitution rule) avoids the problem?\\
\bigskip
\end{flushleft}

\subsection*{The church-Rosse property a.k.a the Diamond property}
\begin{flushleft}
If a lambda- sentence M can be transformed by chains of B transformations into sentences M1 and M2, then there exists (possibly empty) chains of B transformations that transform M1 and M2 into some sentence M3.
\end{flushleft}
\begin{flushleft}
A normal form is a lambda - sentence in which no B transformation is possible \\
If Church-Rosser applies every sentence has \underline{at most} \underline{one} normal form.
\end{flushleft}
\begin{flushleft}
\textbf{\underline{Proof}} Suppose M has 2 normal forms, M1 and M2. But Church Rosser tells us that these are chains of B      transformations that transform M1 and M2 into M3. But M1, M2 are normal forms, so these chains must be empty. So M1=M2=M3.
\end{flushleft}
\begin{flushleft}
(($\lambda$ x y) (($\lambda$ x (x x)) ($\lambda$ x (x x))))
\end{flushleft}
\end{flushleft}
\end{document}