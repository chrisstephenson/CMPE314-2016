\documentclass{article}
\usepackage[margin=2cm,bottom=2cm]{geometry}
\usepackage{hyperref}
\usepackage{comment}
\usepackage[utf8]{inputenc}
\usepackage{graphicx}
\usepackage{mathtools}
\usepackage[normalem]{ulem}
\usepackage{setspace}
%\usepackage{MnSymbol}
\usepackage{soul}

\newcommand\tab[1][1cm]{\hspace*{#1}}
\DeclareMathSizes{10}{10}{10}{10}

\begin{document}
\title{COMP/CMPE 314 - Principles of Programming Languages - Notes}
\author{Chris Stephenson, Istanbul Bilgi University, Department of Mathematics, and course students}
\maketitle
Use map to write a Haskell function that multiplies every element of a possibly infinite list of integers by n.\\ \\
\textbf{Contract}\\
Mullist:: Integer$\rightarrow$ [Integer]$\rightarrow$ [Integer]\\
\textbf{Description}\\
 Mullist n l= map((*) n) l\\
\tab[2.5cm] ($\lambda$ x x*n) l\\ \\
This must be a function that takes an Integer and multiplies it by n\\ \\
Type signature of map?\\
 map:(x$\rightarrow$ y)$\rightarrow$[x] $\rightarrow$ [y]\\ \\
What type signature * ?\\
Integer$\rightarrow$ Integer$\rightarrow$ Integer\\

\section{Exercise 3.59 in SICP}
Infinite series  $ {a}_{0} + {a}_{1}x + {a}_{2}{x}^{2} + {a}_{3}{x}^{3} +.... $ \\ \\
Treat an infite series as an infinite list[a0,a1..]\\ \\
Consider $\int S dx$ .Can we write an integrate function?\\
\textbf{Contract}
[number] $\rightarrow$ [number]\\
$\int$   $ {a}_{0} + {a}_{1}x + {a}_{2}{x}^{2} + {a}_{3}{x}^{3} +....$ dx \\
\tab=k+a0x+a1x2/2+a2x3/3+a3x4/4.... \\ \\
integrate s=1:zipWith\\
integrate s=1:zipWith(/)s[1,2...]\\ \\
Consider a function such that $\int f(x)dx$ = f(x)\\ \\
funny=integrate funny\\
sum s= head s:map((+)(head s))(sums(tail s))\\
powers x=1: map(* x)(powers x)\\

\cleardoublepage
\section{The Most Beautiful Function}

(LET ((x ($\lambda$x x)) (succ($\lambda$ n($\lambda$ f($\lambda$ x (f ((n f) x)))))\\

(LET n(x ($\lambda$x x))\\

(LET n(succ...))\\
\tab$< body >$)\\

(LET n(v $< expr >$)) $< body >$)\\

\tab(($\lambda$ $\upsilon$ $< body >$) $< expr >$)\\

we want to write factorial\\

(LET(fac($\lambda$ n(((IF (ZERO? n)) ONE)(fac(PRED n)))))\\

All defined in LETS\\

(($\lambda$ fac(fac SEVEN))($\lambda$ n((IF(ZERO? n)ONE)(fac(PRED n))))\\

Our factorial function needs to look like this\\
\tab($\lambda$ fac($\lambda$ n....fac....))\\

Then we need to make a closure of fac applied to itself\\

\tab(($\lambda$ f (($\lambda$ x(f(x x)))($\lambda$ x(f(x x))))) F)\\

beta transform$\rightarrow$(($\lambda$ x (F(x x))($\lambda$ x(F(x x)))) = (YF)\\

\tab(F(($\lambda$ x(F (x x))($\lambda$ x (F (x x))))=F(YF)\\

The Y combinator\\    

\end{document}