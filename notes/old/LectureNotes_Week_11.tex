\documentclass{article}
\usepackage[margin=2cm,bottom=2cm]{geometry}
\usepackage{hyperref}
\usepackage{comment}
\usepackage[utf8]{inputenc}
\usepackage{mathtools}
\usepackage{color}
\usepackage{amsmath}
\usepackage{array}
\usepackage{setspace}
\usepackage{soul}
\usepackage{amssymb}
\usepackage{graphicx}
\usepackage{tcolorbox}
\usepackage{color}
\newcommand{\tab}[1]{\hspace{0.12\textwidth}\rlap{#1}}

\begin{document}
\title{COMP/CMPE 314 - Principles of Programming Languages - Notes}
\author{Chris Stephenson, Istanbul Bilgi University, Department of Mathematics, and course students}
\date{May,5}
\maketitle

\section*{Using Map for Haskell Function}
\begin{flushleft}
Use map to write a Haskell function that multiplies every element of a possibly infinite list of integers by n.\\
\bigskip

{\large $\bigstar$ mulList :: Integer $\rightarrow$ [Integer] $\rightarrow$ [Integer]\\
\vspace*{0.2cm}
$\bigstar$ mulList n l= map ((*) n) \underline{$\ell$}}\\
\bigskip
{\LARGE \tab{$\downarrow$}}\\
\bigskip
This must be a function that takes an integer and multiplies it by n.\\
\vspace*{0.8cm}
\emph{\begin{large}
Type signature of map?\\
$\succ$ map :: (x $\rightarrow$ y) $\rightarrow$ [x] $\rightarrow$ [y]\\
\bigskip
Type signature of x ?\\
$\succ$ Integer $\rightarrow$ Integer $\rightarrow$ Integer\\
\end{large}}
\bigskip
\vspace*{1cm}
{\large \textbf{\underline{Exercise 3.59} in SICP}}\\
\vspace*{0.5cm}
{\normalsize \textbf{\underline{Infinite series}}}\\
\vspace*{0.5cm}
Infinite series  \emph{{\large S}= $ {a}_{0} + {a}_{1}x + {a}_{2}{x}^{2} + {a}_{3}{x}^{3} +\ldots $} treat an infinite series as an infinite list \emph{[a0,a1...]}\\
\bigskip
\hspace{0.9cm}\textbf{Consider} {\large $\int_{y}^{x} S dx$}. Can we write an \underline{integrate} function?\\
\bigskip
\hspace{0.9cm}\textbf{Contract}
[Number] $\rightarrow$ [Number]\\
\begin{large}
$$\int_{0}^{x} {a}_{0} + {a}_{1}x + {a}_{2}{x}^{2} + {a}_{3}{x}^{3} +\ldots dx $$ \\
$$= {a}_{0}x + {a}_{1}\dfrac{x^{2}}{2} + {a}_{2}\dfrac{x^{3}}{3} + {a}_{3}\dfrac{x^{4}}{4}$$
\end{large} 
\cleardoublepage
\begin{flushleft}
\emph{Integrate s = 1 : zipWith\\
\bigskip
Consider a function such that $\int f(x)dx$ = f(x)\\
\tab funny = integrate funny\\
\bigskip
powers x = 1 : map($\ast$ x) (powers x)}\\
\vspace*{1cm}
\textbf{\underline{ZipWith Signature: }}\\
\bigskip
(a $\rightarrow$ b $\rightarrow$ c) $\rightarrow$ [a] $\rightarrow$ [b] $\rightarrow$ [c]
\end{flushleft}
\bigskip
\subsection*{The most beautiful function}
\begin{flushleft}
(LET n ($\nu$ $< expr >$) $< body >$)\\
\bigskip
{\LARGE \tab{$\downarrow$ desugar }}\\
\bigskip
(($\lambda$ $\nu$ $< body >$) $< expr >$)\\
\bigskip
\textcolor{red}{(LET (fac} ($\lambda$ n (((\underline{IF} (\underline{ZERO?} n)) \underline{ONE}) (\underline{fac} (\underline{PRED} n)))))\\
\bigskip
{\LARGE \tab{$\swarrow\downarrow\searrow$}}\\
\bigskip
\textcolor{red}{(fac SEVEN) all defined in LETS}\\
\bigskip

{\LARGE \tab{$\Downarrow$ desugar}}\\
\bigskip
(($\lambda$ fac (fac SEVEN)) ($\lambda$ n ((IF (ZERO? n) ONE) (\textcolor{red}{fac} (PRED  n))))\\

\vspace*{1cm}
Our factorial function needs to look like this: \\
( $\lambda$ fac ($\lambda$ n ..... fac .....))\\
\vspace*{1cm}
Then we need to make a closure of fac applied to \underline{itself}\\
\vspace*{0.4cm}
(\underline{($\lambda$ f (($\lambda$ x (f (x x))) ($\lambda$ x (f (x x)))))} F)\\
\textcolor{red}{\tab{y}}\\
\bigskip
{\LARGE \tab{$\downarrow$ $\beta$}}\\
\bigskip
(($\lambda$ x (F (x x)) ($\lambda$ x (F (x x))))\\
\bigskip
{\LARGE \tab{$\downarrow$ $\beta$}}\\
\bigskip
(F (($\lambda$ x (F (x x)) ($\lambda$ x (F (x x)))) $\equiv$ F(YF)\\
\vspace*{1.5cm}
The \underline{Y combinator}  




	
\end{flushleft}
\end{flushleft}
\end{document}